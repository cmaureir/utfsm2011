Como hemos podido darnos cuenta, los distintos
enfoques de software que son requeridos en el área
de la astronomía, están caracterizados por la apariencia
de las interfaces.

Por el caso de los simuladores
y observatorios virtuales, el diseño
es muy similar, conteniendo una vista principal del entorno
simulado, con un menú de herramientas en la parte superior.

Por otro lado, es necesario buscar
soluciones para software que necesiten configurar
demasiada información, ya que un mal diseño
puede llegar a una total inutilización de una interfaz
debido a la enormidad de información critica en un solo lugar.

Finalmente, es necesario recalcar que no existen
benchmark de fácil acceso sobre el desarrollo
de interfaces astronómicas generalizadas,
por lo tanto se puede considerar esta área,
como un nicho de investigación para realizar el aporte
a la comunidad científica.
