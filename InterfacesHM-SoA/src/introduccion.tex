El desarrollo de interfaces a nivel de proyectos astronómicos,
ha sido un factor clave a la hora de realizar un buen trabajo,
ya sea en el área de control automatizado, análisis de datos,
configuraciones, observatorios virtuales, etc.

Si bien es cierto, el desarrollo de una interfaz gráfica de usuario
no es un trabajo fácil, esta complejidad aumenta debido a que las
funcionalidades están sujetas a procesos netamente de una área
en particular, como pueden ser física, matemáticas, electrónica,
etc, creando así nuevas metas que requerirán cierta validación
a la hora de programarlas, como por ejemplo no perder precisión
de datos obtenidos de un dispotivo electrónico determinado,
pero a la misma vez realizando un cálculo eficiente.

