\section{Primera pregunta}

Una nueva empresa desea vender un producto, en donde el estudio de mercado,
arrojó como resultado que la firma está en condiciones:

\begin{center}
    \begin{tabular}{|c|c|}
        \hline
        Año   &  Demanda\\\hline
        2006  &  60\\
        2007  &  76\\
        2008  &  88\\
        2009  &  97\\
        2010  &  111\\\hline
    \end{tabular}
\end{center}

La empresa planea ingresar en el mercado el año 2012 Según el análisis de precios,
se determinó que para el primer año de ejercicio del proyecto, el precio unitario
por cada producto es de \$110.000, el cual se proyecta con un crecimiento real anual de un 8\%.

Dado el crecimiento que proyecta que experimentarán las ventas de este servicio,
usted ha determinado que los costos operacionales totales serán de un 60\% de las ventas.
En cuanto a las remuneraciones, se calcula que el total de las remuneraciones brutas para
el primer año es de \$3.500.000, siendo el aumento real anual de las remuneraciones de un 2\%. 

La inversión en capital fijo (depreciable) es \$6.000.000 netos, con vida útil normal según SII de 3 años.
Considerar que se deprecian linealmente, y que no hay reinversiones en el horizonte de evaluación.
Por otro lado, la inversión en intangibles (formación de sociedad, gastos de puesta en marcha, marketing, etc.),
se ha estimado en \$1.500.000 
Las condiciones de financiamiento para este nuevo producto,
es de un 80\% de la inversión con crédito externo al 5\%, con amortizaciones iguales y pagaderos en 2 años. 

El horizonte de evaluación es de 3 años y una tasa de descuento del 15\%.
La empresa piensa cerrar en el horizonte del proyecto, y no considera vender los activos.

\textbf{Nota:} Considerar el período de desfase para la inversión de capital de trabajo como 6 meses.

Se pide:
\newpage
\begin{enumerate}[(a)]

    \item {\bf Proyección de Demanda (2012 - 2014) (8p)}

          \red{Respuesta:}

          Considerando los valores de la demanda entre los años 2006 y 2010 señalados
          en el enunciado, se procede a obtener los valores necesarios para obtener la ecuación
          de proyección de demanda.

          \begin{eqnarray}
              y = b + a\cdot x 
          \end{eqnarray}

          Siendo, $y$ la demanda (variable dependiente), $x$ variable independiente,
          $a$ el punto de intersección de la línea de regresión con el eje $Y$,
          $b$ la pendiente de la línea de regresión.

          Para calcular tanto $a$, como $b$ existen las siguientes fórmulas:

          \begin{eqnarray}
              b &=& \dfrac{\sum_{i=1}^{n} (x\cdot y) - n\cdot \bar{x}\cdot \bar{y}}{\sum_{i=1}^{n} x_{t}^{2} - n\cdot \bar{x}^{2}} \\
              a &=& \bar{y} - b\cdot \bar{x} 
          \end{eqnarray}

          Podemos obtener los valores de las fórmulas de la siguiente forma:

          \begin{table}[h!t]
            \centering
            \begin{tabular}{|c|r|r|r|r|r|}
                \hline
                {\bf Año}      & {\bf $x$} & {\bf $y$} & {\bf $x\cdot y$} & {\bf $x^{2}$} & {\bf $y^{2}$} \\ \hline
                2006           & -2      & 60      & -120           & 4             & 3600          \\
                2007           & -1      & 76      & -76            & 1             & 5776          \\
                2008           & 0       & 88      & 0              & 0             & 7744          \\
                2009           & 1       & 97      & 97             & 1             & 9409          \\
                2010           & 2       & 111     & 222            & 4             & 12321         \\ \hline
                {\bf Suma}     & 0       & 432     & 123            & 10            & 38850         \\ \hline
                {\bf Promedio} & 0       & 86.4    & 24.6           &               &               \\ \hline
            \end{tabular}
          \end{table}            

          Reemplazando valores en las fórmulas:
    
          \begin{eqnarray}
              b &=& \frac{123 - 0}{10 - 0} \nonumber \\
                &=& 12.3  \\
              a &=& 86.4 - 12.3\cdot 0 \nonumber \\
                &=& 86.4
          \end{eqnarray}

          Obtenemos finalmente la fórmula general:

          \begin{eqnarray}
              y = 12.3 + 86.4\cdot x 
          \end{eqnarray}

          Ahora es posible realizar la \emph{proyección de la demanda} entre los años 2012 y 2014.
        
          \begin{table}[h!t]
            \centering
            \begin{tabular}{|c|r|r|r|r|r|}
                \hline
                {\bf Año} & {\bf x} & {\bf y} & {\bf $\approx y$}\\ \hline
                2012      & 4       & 135.6   & 136  \\
                2013      & 5       & 147.9   & 148  \\
                2014      & 6       & 160.2   & 161  \\ \hline
            \end{tabular}
            \caption{Proyección de la demanda}
            \label{tab:demanda}
          \end{table}            

        \begin{center}
        \begin{tikzpicture}
            \begin{axis}[
                xlabel=$\text{Año} (x)$,
                ylabel=$\text{Demanda} (y)$]
            \addplot[smooth,mark=*,blue] plot coordinates {
                (2006,60)
                (2007,76)
                (2008,88)
                (2009,97)
                (2010,111)
                (2012,136)
                (2013,148)
                (2014,161)
            };
            \end{axis}
        \end{tikzpicture}
        \end{center}

\newpage
    \item {\bf Tabla de Inversiones (6p)}

          \red{Respuesta:}

        Del enunciado podemos obtener el valor de la inversión del \emph{capital fijo} y \emph{capital intangible},
        pero es necesario obtener la inversión del \emph{capital de trabajo}, utilizando la siguiente fórmula:

        \begin{eqnarray}
            \text{Inversión Capital Trabajo} &=& \frac{\text{Periodo de desfase}}{\text{12 meses (1 año)}} \cdot \text{Egresos primer año}
        \end{eqnarray}

        De la misma forma,
        los \emph{egresos del primer año} se pueden obtener sumando los \emph{costos operacionales totales} mas las \emph{remuneraciones}.
        \begin{eqnarray}
            \text{Egresos primer año} &=& \text{Costos operacionales totales} \cdot \text{Remuneraciones} \\
                                      &=& 0.6\cdot  \text{Ventas} \cdot \text{Remuneraciones} \nonumber \\
                                      &=& 12476000 \nonumber
        \end{eqnarray}        
        Por lo tanto la \emph{inversión de capital de trabajo} es de $\$6,238,000$.

        Quedando así las siguientes inversiones:

        \begin{table}[h!t]
            \centering
            \begin{tabular}{|l|r|}
                \hline
                {\bf Inversión }   & {\bf Monto} \\ \hline
                Capital de Trabajo & \$6,238,000     \\ \hline
                Capital Fijo       & \$6,000,000     \\ \hline
                Capital Intangible & \$1,500,000     \\ \hline
                {\bf Total inversión primer año} & \$13,738,000 \\\hline
            \end{tabular}
            \caption{Tabla de inversiones}
            \label{tab:inversiones}
        \end{table}

    \item {\bf Tabla de Depreciación (5p)}

          \red{Respuesta:}

          Considerando un capital fijo de $\$6,000,000$,
          y aplicando una depreciación lineal,
          con una vida útil de 3 años, podemos establecer lo siguiente.

          \begin{table}[h!t]
            \centering
            \begin{tabular}{|c|r|r|}
                \hline
                {\bf Periodo} & {\bf Depreciación} & {\bf Valor activo} \\\hline
                0 & - & 6,000,000 \\\hline
                1 & 2,000,000  & 4,000,000  \\\hline
                2 & 2,000,000  & 2,000,000  \\\hline
                3 & 2,000,000  & 0  \\\hline
            \end{tabular}
            \caption{Tabla de depreciación}
            \label{tab:depreciacion}  
          \end{table}

\newpage
    \item {\bf Tabla de Amortización (6p)}

          \red{Respuesta:}

        Considerando la información entregada en el enunciado,
        que señala que el financiamiento con crédito externo
        es sólo el 80\% de la inversión total,  con un interés de 5\%
        y utilizando el enfoque de amortización igual en 2 años,
        podemos obtener los siguientes datos:

        \begin{eqnarray}
            \text{Principal inicial} &=& \text{Inversión total}\cdot 0.8 \\
                                     &=& 10990400 \nonumber \\
            \text{Amortización fija} &=& \dfrac{\text{Principal inicial}}{\text{periodo de duración}} \\
                                     &=& \dfrac{10990400}{2} \nonumber \\
                                     &=& 5495200 \nonumber 
        \end{eqnarray}

        Además, se irá actualizando la tabla considerando que:

        \begin{eqnarray}
            \text{Interés} &=& \text{Principal}\cdot 0.05 \\
            \text{Cuota}   &=& \text{Amortización fija} + \text{Interés} 
        \end{eqnarray}

        \begin{table}[h!t]
            \centering
            \begin{tabular}{|c|c|c|c|c|}
                \hline
                {\bf Periodo} & {\bf Principal }   & {\bf Amortización} & {\bf Interés} & {\bf Cuota} \\\hline
                0             & 10990400           &        -           & -                  & -      \\\hline
                1             & 5495200            &  5495200           & 549520             & 6044720\\\hline
                2             & 0                  &  5495200           & 274760             & 5769960\\\hline
            \end{tabular}
            \caption{Tabla de Amortización}
            \label{tab:amortizacion}          
        \end{table}




\newpage
    \item {\bf Flujo de Caja Financiado con un horizonte de evaluación de 3 años (20p)}

          \red{Respuesta:}
          \begin{table}[h!t]
                \centering
                \footnotesize                
                \begin{tabular}{|l|r|r|r|r|}
                    \hline
                    {\bf Item}                               & {\bf Año 0}   & {\bf Año 1 }  & {\bf Año 2}   & {\bf Año 3}   \\\hline
                    {    Ingresos (+)}                       &               & 14,960,000.00 & 17,582,400.00 & 20,656,944.00 \\\hline
                    {    Costos operacionales fijos (-)}     &               &  8,976,000.00 & 10,549,440.00 & 12,394,166.40 \\\hline
                    {    Remuneraciones (-)}                 &               &  3,500,000.00 &  3,570,000.00 &  3,641,400.00 \\\hline
                    {\bf Utilidad operacional (=)}           &               &{\bf 2,484,000.00} &{\bf 3,462,960.00} &{\bf 4,621,377.60} \\\hline
                    {    Depreciación (-)}                   &               &  2,000,000.00 &  2,000,000.00 &  2,000,000.00 \\\hline
                    {    Perdida ejercicio anterior (-)}     &               &               &    -65,520.00 &               \\\hline
                    {    Intereses crédito (-)}              &               &    549,520.00 &    274,760.00 &               \\\hline
                    {\bf Utilidad antes impuesto (=)}        &               &{\bf-65,520.00}&{\bf 1,122,680.00} &{\bf 2,621,377.60} \\\hline
                    {    Impuestos 17\% (-)}                 &               &          0.00 &    190,855.60 &    445,634.19 \\\hline
                    {\bf Utilidad despues impuesto (=)}      &               &{\bf-65,520.00}&{\bf 931,824.40}&{\bf 2,175,743.41} \\\hline
                    {    Depreciación (+)}                   &               &  2,000,000.00 &  2,000,000.00 &  2,000,000.00 \\\hline
                    {    Inversión (-)}                      & 13738000      &               &               &               \\\hline
                    {    Amortización (-)}                   &               &  5,495,200.00 &  5,495,200.00 &               \\\hline
                    {    Perdida ejercicio anterior (+)}     &               &               &    -65,520.00 &               \\\hline
                    {    Créditos (+)}                       & 10990400      &               &               &               \\\hline
                    {    Recuperación capital de trabajo (+)}&               &               &               &  6,238,000.00 \\\hline
                    { \blue{Flujo de Caja} }              &\blue{ -2,747,600.00} &\blue{ -3,560,720.00} &\blue{ -2,497,855.60} & \blue{10,413,743.41} \\\hline
                    
                \end{tabular}
                
            \end{table}
\newpage
    \item {\bf Indicadores Económicos  del Flujo de Caja Financiado (VAN, PAYBACK, TIR y TIRM) (10p)}

          \red{Respuesta:}

          lorem ipsum

\newpage
    \item {\bf Realizar la sensibilización con las variables “ingreso” e “inversiones” (10p)}

          \red{Respuesta:}

          lorem ipsum

\newpage
    \item {\bf Conclusiones, analizando detalladamente la información obtenida del flujo de caja y de los indicadores económico? (10p)}

          \red{Respuesta:}

          lorem ipsum

\end{enumerate}
