\section{Primera pregunta}

Una nueva empresa desea vender un producto, en donde el estudio de mercado,
arrojó como resultado que la firma está en condiciones:

\begin{center}
    \begin{tabular}{|c|c|}
        \hline
        Año   &  Demanda\\\hline
        2006  &  60\\
        2007  &  76\\
        2008  &  88\\
        2009  &  97\\
        2010  &  111\\\hline
    \end{tabular}
\end{center}

La empresa planea ingresar en el mercado el año 2012 Según el análisis de precios,
se determinó que para el primer año de ejercicio del proyecto, el precio unitario
por cada producto es de \$110.000, el cual se proyecta con un crecimiento real anual de un 8\%.

Dado el crecimiento que proyecta que experimentarán las ventas de este servicio,
usted ha determinado que los costos operacionales totales serán de un 60\% de las ventas.
En cuanto a las remuneraciones, se calcula que el total de las remuneraciones brutas para
el primer año es de \$3.500.000, siendo el aumento real anual de las remuneraciones de un 2\%. 

La inversión en capital fijo (depreciable) es \$6.000.000 netos, con vida útil normal según SII de 3 años.
Considerar que se deprecian linealmente, y que no hay reinversiones en el horizonte de evaluación.
Por otro lado, la inversión en intangibles (formación de sociedad, gastos de puesta en marcha, marketing, etc.),
se ha estimado en \$1.500.000 
Las condiciones de financiamiento para este nuevo producto,
es de un 80\% de la inversión con crédito externo al 5\%, con amortizaciones iguales y pagaderos en 2 años. 

El horizonte de evaluación es de 3 años y una tasa de descuento del 15\%.
La empresa piensa cerrar en el horizonte del proyecto, y no considera vender los activos.

\textbf{Nota:} Considerar el período de desfase para la inversión de capital de trabajo como 6 meses.

Se pide:
\newpage
\begin{enumerate}[(a)]

    \item {\bf Proyección de Demanda (2012 - 2014) (8p)}

          \red{Respuesta:}

          Considerando los valores de la demanda entre los años 2006 y 2010 señalados
          en el enunciado, se procede a obtener los valores necesarios para obtener la ecuación
          de proyección de demanda.

          \begin{eqnarray}
              y = b + a\cdot x 
          \end{eqnarray}

          Siendo, $y$ la demanda (variable dependiente), $x$ variable independiente,
          $a$ el punto de intersección de la línea de regresión con el eje $Y$,
          $b$ la pendiente de la línea de regresión.

          Para calcular tanto $a$, como $b$ existen las siguientes fórmulas:

          \begin{eqnarray}
              b &=& \dfrac{\sum_{i=1}^{n} (x\cdot y) - n\cdot \bar{x}\cdot \bar{y}}{\sum_{i=1}^{n} x_{t}^{2} - n\cdot \bar{x}^{2}} \\
              a &=& \bar{y} - b\cdot \bar{x} 
          \end{eqnarray}

          Podemos obtener los valores de las fórmulas de la siguiente forma:

          \begin{table}[h!t]
            \centering
            \begin{tabular}{|c|r|r|r|r|r|}
                \hline
                {\bf Año}      & {\bf $x$} & {\bf $y$} & {\bf $x\cdot y$} & {\bf $x^{2}$} & {\bf $y^{2}$} \\ \hline
                2006           & -2      & 60      & -120           & 4             & 3600          \\
                2007           & -1      & 76      & -76            & 1             & 5776          \\
                2008           & 0       & 88      & 0              & 0             & 7744          \\
                2009           & 1       & 97      & 97             & 1             & 9409          \\
                2010           & 2       & 111     & 222            & 4             & 12321         \\ \hline
                {\bf Suma}     & 0       & 432     & 123            & 10            & 38850         \\ \hline
                {\bf Promedio} & 0       & 86.4    & 24.6           &               &               \\ \hline
            \end{tabular}
          \end{table}            

          Reemplazando valores en las fórmulas:
    
          \begin{eqnarray}
              b &=& \frac{123 - 0}{10 - 0} \nonumber \\
                &=& 12.3  \\
              a &=& 86.4 - 12.3\cdot 0 \nonumber \\
                &=& 86.4
          \end{eqnarray}

          Obtenemos finalmente la fórmula general:

          \begin{eqnarray}
              y = 12.3 + 86.4\cdot x 
          \end{eqnarray}

          Ahora es posible realizar la \emph{proyección de la demanda} entre los años 2012 y 2014.
        
          \begin{table}[h!t]
            \centering
            \begin{tabular}{|c|r|r|r|r|r|}
                \hline
                {\bf Año} & {\bf x} & {\bf y} & {\bf $\approx y$}\\ \hline
                2012      & 4       & 135.6   & 136  \\
                2013      & 5       & 147.9   & 148  \\
                2014      & 6       & 160.2   & 161  \\ \hline
            \end{tabular}
            \caption{Proyección de la demanda}
            \label{tab:demanda}
          \end{table}            

        \begin{center}
        \begin{tikzpicture}
            \begin{axis}[
                xlabel=$\text{Año} (x)$,
                ylabel=$\text{Demanda} (y)$]
            \addplot[smooth,mark=*,blue] plot coordinates {
                (2006,60)
                (2007,76)
                (2008,88)
                (2009,97)
                (2010,111)
                (2012,136)
                (2013,148)
                (2014,161)
            };
            \end{axis}
        \end{tikzpicture}
        \end{center}

\newpage
    \item {\bf Tabla de Inversiones (6p)}

          \red{Respuesta:}

        Del enunciado podemos obtener el valor de la inversión del \emph{capital fijo} y \emph{capital intangible},
        pero es necesario obtener la inversión del \emph{capital de trabajo}, utilizando la siguiente fórmula:

        \begin{eqnarray}
            \text{Inversión Capital Trabajo} &=& \frac{\text{Periodo de desfase}}{\text{12 meses (1 año)}} \cdot \text{Egresos primer año}
        \end{eqnarray}

        De la misma forma,
        los \emph{egresos del primer año} se pueden obtener sumando los \emph{costos operacionales totales} mas las \emph{remuneraciones}.
        \begin{eqnarray}
            \text{Egresos primer año} &=& \text{Costos operacionales totales} \cdot \text{Remuneraciones} \\
                                      &=& 0.6\cdot  \text{Ventas} \cdot \text{Remuneraciones} \nonumber \\
                                      &=& 12476000 \nonumber
        \end{eqnarray}        
        Por lo tanto la \emph{inversión de capital de trabajo} es de $\$6,238,000$.

        Quedando así las siguientes inversiones:

        \begin{table}[h!t]
            \centering
            \begin{tabular}{|l|r|}
                \hline
                {\bf Inversión }   & {\bf Monto} \\ \hline
                Capital de Trabajo & \$6,238,000     \\ \hline
                Capital Fijo       & \$6,000,000     \\ \hline
                Capital Intangible & \$1,500,000     \\ \hline
                {\bf Total inversión primer año} & \$13,738,000 \\\hline
            \end{tabular}
            \caption{Tabla de inversiones}
            \label{tab:inversiones}
        \end{table}

    \item {\bf Tabla de Depreciación (5p)}

          \red{Respuesta:}

          Considerando un capital fijo de $\$6,000,000$,
          y aplicando una depreciación lineal,
          con una vida útil de 3 años, podemos establecer lo siguiente.

          \begin{table}[h!t]
            \centering
            \begin{tabular}{|c|r|r|}
                \hline
                {\bf Periodo} & {\bf Depreciación} & {\bf Valor activo} \\\hline
                0 & - & 6,000,000 \\\hline
                1 & 2,000,000  & 4,000,000  \\\hline
                2 & 2,000,000  & 2,000,000  \\\hline
                3 & 2,000,000  & 0  \\\hline
            \end{tabular}
            \caption{Tabla de depreciación}
            \label{tab:depreciacion}  
          \end{table}

\newpage
    \item {\bf Tabla de Amortización (6p)}

          \red{Respuesta:}

        Considerando la información entregada en el enunciado,
        que señala que el financiamiento con crédito externo
        es sólo el 80\% de la inversión total,  con un interés de 5\%
        y utilizando el enfoque de amortización igual en 2 años,
        podemos obtener los siguientes datos:

        \begin{eqnarray}
            \text{Principal inicial} &=& \text{Inversión total}\cdot 0.8 \\
                                     &=& 10990400 \nonumber \\
            \text{Amortización fija} &=& \dfrac{\text{Principal inicial}}{\text{periodo de duración}} \\
                                     &=& \dfrac{10990400}{2} \nonumber \\
                                     &=& 5495200 \nonumber 
        \end{eqnarray}

        Además, se irá actualizando la tabla considerando que:

        \begin{eqnarray}
            \text{Interés} &=& \text{Principal}\cdot 0.05 \\
            \text{Cuota}   &=& \text{Amortización fija} + \text{Interés} 
        \end{eqnarray}

        \begin{table}[h!t]
            \centering
            \begin{tabular}{|c|c|c|c|c|}
                \hline
                {\bf Periodo} & {\bf Principal }   & {\bf Amortización} & {\bf Interés} & {\bf Cuota} \\\hline
                0             & 10990400           &        -           & -                  & -      \\\hline
                1             & 5495200            &  5495200           & 549520             & 6044720\\\hline
                2             & 0                  &  5495200           & 274760             & 5769960\\\hline
            \end{tabular}
            \caption{Tabla de Amortización}
            \label{tab:amortizacion}          
        \end{table}




\newpage
    \item {\bf Flujo de Caja Financiado con un horizonte de evaluación de 3 años (20p)}

          \red{Respuesta:}
          \begin{table}[h!t]
                \centering
                \footnotesize                
                \begin{tabular}{|l|r|r|r|r|}
                    \hline
                    {\bf Item}                               & {\bf Año 0}   & {\bf Año 1 }  & {\bf Año 2}   & {\bf Año 3}   \\\hline
                    {    Ingresos (+)}                       &               & 14,960,000.00 & 17,582,400.00 & 20,656,944.00 \\\hline
                    {    Costos operacionales fijos (-)}     &               &  8,976,000.00 & 10,549,440.00 & 12,394,166.40 \\\hline
                    {    Remuneraciones (-)}                 &               &  3,500,000.00 &  3,570,000.00 &  3,641,400.00 \\\hline
                    {\bf Utilidad operacional (=)}           &               &{\bf 2,484,000.00} &{\bf 3,462,960.00} &{\bf 4,621,377.60} \\\hline
                    {    Depreciación (-)}                   &               &  2,000,000.00 &  2,000,000.00 &  2,000,000.00 \\\hline
                    {    Perdida ejercicio anterior (-)}     &               &               &    -65,520.00 &               \\\hline
                    {    Intereses crédito (-)}              &               &    549,520.00 &    274,760.00 &               \\\hline
                    {\bf Utilidad antes impuesto (=)}        &               &{\bf-65,520.00}&{\bf 1,122,680.00} &{\bf 2,621,377.60} \\\hline
                    {    Impuestos 17\% (-)}                 &               &          0.00 &    190,855.60 &    445,634.19 \\\hline
                    {\bf Utilidad después impuesto (=)}      &               &{\bf-65,520.00}&{\bf 931,824.40}&{\bf 2,175,743.41} \\\hline
                    {    Depreciación (+)}                   &               &  2,000,000.00 &  2,000,000.00 &  2,000,000.00 \\\hline
                    {    Inversión (-)}                      & 13738000      &               &               &               \\\hline
                    {    Amortización (-)}                   &               &  5,495,200.00 &  5,495,200.00 &               \\\hline
                    {    Perdida ejercicio anterior (+)}     &               &               &    -65,520.00 &               \\\hline
                    {    Créditos (+)}                       & 10990400      &               &               &               \\\hline
                    {    Recuperación capital de trabajo (+)}&               &               &               &  6,238,000.00 \\\hline
                    { \blue{Flujo de Caja} }              &\blue{ -2,747,600.00} &\blue{ -3,560,720.00} &\blue{ -2,497,855.60} & \blue{10,413,743.41} \\\hline
                    
                \end{tabular}
                
            \end{table}
\newpage
    \item {\bf Indicadores Económicos  del Flujo de Caja Financiado (VAN, PAYBACK, TIR y TIRM) (10p)}

          \red{Respuesta:}

            \begin{itemize}
                \item {\bf VAN}\\

                    El Valor Actual Neto (VAN), nos permite poder determinar el valor presente, considerando un número de flujos de caja del futuro de un proyecto,
                    originados por una inversión inicial.

                    Para poder calcularlo, debemos utilizar la siguiente fórmula:

                    \begin{eqnarray}
                        \mbox{VAN} = \sum_{t=1}^n{\frac{V_t}{(1+k)^t}}- I_0
                    \end{eqnarray}

                    Siendo sus variables:
                    \begin{itemize}
                        \item $V_{t}$, flujos de caja del periodo $t$.
                        \item $I_{0}$, valor de la inversión inicial.
                        \item $n$, número de periodos considerados.
                        \item $k$, tipo de interés.
                    \end{itemize}

                    El valor obtenido con los valores del flujo de caja de la pregunta anterior es:

                    $$VAN = -\$885,410.62$$

                \item {\bf PAYBACK}\\

                    El PAYBACK se denomina también como plazo de recuperación,
                    consiste en uno de los métodos de la selección estática,
                    la cual consiste en poder tener una idea aproximada
                    de lo que se tardará un proyecto en poder recuperar
                    el desembolso inicial en el proyecto.


                   En esencia, iremos verificando periodo a periodo el VAN, lo cual nos dirá si la cantidad obtenida
                   es positiva o negativa. En caso de ser negativa, significa que no hemos  recuperado la inversión
                   y análogamente, siendo positiva, hemos logrado recuperar la inversión.


                   Los valores obtenidos para cada periodo, utilizando los resultados del flujo de caja fueron:

                   \begin{table}[h!t]
                        \centering
                        \begin{tabular}{|c|c|}
                            \hline
                            {\bf Periodo} & Valor Payback \\\hline
                            1 & -5,843,878.00 \\
                            2 & -7,732,616.00 \\
                            3 &   -885,411.00 \\ \hline
                        \end{tabular}
                   \end{table}

                \item {\bf TIR}\\

                    La Tasa Interna de Retorno (TIR), está definida como la tasa de interés necesaria para que el Valor Actual Neto (VAN) sea igual a cero.

                    La fórmula para poder obtenerla es similar a la del VAN, sólo que ahora el valor $k$ será nuestra incógnita, es decir:

                    \begin{eqnarray}
                         VAN = \sum_{t=1}^n{\frac{F_{t}}{(1+TIR)^t}} -I = 0
                    \end{eqnarray}

                    El valor obtenido con los resultados del flujo de caja es de:
            
                    $$TIR = 0.0851 \approx 9\%$$

                \item {\bf TIRM}\\

                    La Tasa Interna de Retorno Modificada (TIRM), nace para poder acercar a la realidad el valor obtenido mediante la TIR,
                    pues dicha tasa no nos entrega un valor muy preciso.

                    La fórmula del TIRM está definida como:
            
                    \begin{eqnarray}
                        TIRM = \sqrt[n]{\frac{VAN}{I_0}-1}\cdot (1 + k) -1 
                    \end{eqnarray}

                    Siendo sus variables:
                    \begin{itemize}
                        \item $VAN$, Valor Actual Neto.
                        \item $I_{0}$, valor de la inversión inicial.
                        \item $n$, número de periodos considerados.
                        \item $k$, tipo de interés.
                    \end{itemize}

                    El valor obtenido utilizando el resultado del flujo de caja es:

                    $$TIRM = 0.124 \approx 12\%$$
            \end{itemize}

\newpage
    \item {\bf Realizar la sensibilización con las variables “ingreso” e “inversiones” (10p)}

          \red{Respuesta:}

          Para ambas sensibilizaciones, se ha considerado el siguiente rango de variación:
          $-20\%, -10\%, -5\%, +5\%, +10\%. +20\%$.

          Además, para medir la sensibilidad de las variables involucradas se utilizará
          una medida $S$, dada por :
          \begin{eqnarray}
              S = \|\dfrac{VAN_{\text{nuevo}} - VAN_{\text{viejo}}}{VAN_{\text{viejo}}}\|
          \end{eqnarray}
          con la cual podemos  establecer las siguientes condiciones:

          \begin{itemize}
              \item $S > 1$, Proyecto {\bf muy} sensible al cambio de variable
              \item $S < 1$, Proyecto {\bf poco} sensible al cambio de variable
          \end{itemize}

          \begin{itemize}
              \item {\bf Ingresos}\\
                \begin{table}[h!t]
                    \centering
                    \scriptsize
                    \begin{tabular}{|c|r|r|r|r|r|r|}
                        \hline
                        {\bf Ingresos Año}   & -20\%           & -10\%           & -5\%            & +5\%            & +10\%           & +20\% \\\hline
                        2012                 & \$11,968,000.00 & \$13,464,000.00 & \$14,212,000.00 & \$15,708,000.00 & \$16,456,000.00 & \$17,952,000.00 \\ \hline
                        2013                 & \$14,065,920.00 & \$15,824,160.00 & \$16,703,280.00 & \$18,461,520.00 & \$19,340,640.00 & \$21,098,880.00 \\ \hline
                        2014                 & \$16,525,555.20 & \$18,591,249.60 & \$19,624,096.80 & \$21,689,791.20 & \$22,722,638.40 & \$24,788,332.80 \\ \hline
                        \hline
                        {\bf Flujos de caja} & -20\%           & -10\%           & -5\%            & +5\%            & +10\%           & +20\% \\\hline
                        Año 0                & -\$2,568,080.00 & -\$2,657,840.00 & -\$2,702,720.00 & -\$2,792,480.00 & -\$2,837,360.00 & -\$2,927,120.00\\\hline
                        Año 1                & -\$4,362,576.00 & -\$3,961,648.00 & -\$3,761,184.00 & -\$3,360,256.00 & -\$3,159,792.00 & -\$2,758,864.00\\\hline
                        Año 2                & -\$3,094,034.48 & -\$2,795,945.04 & -\$2,646,900.32 & -\$2,348,810.88 & -\$2,199,766.16 & -\$1,901,676.72\\\hline
                        Año 3                & \$8,144,522.33  & \$9,279,132.87  & \$9,846,438.14  & \$10,981,048.68 & \$11,548,353.95 & \$12,682,964.49\\\hline
                        \hline
                        {\bf VAN }           & -\$3,346,003.29 & -\$2,115,706.96 & -\$1,500,558.79 & -\$270,262.46   & \$344,885.71    & \$1,575,182.04 \\\hline
                        {\bf Sensibilidad}   & 2.78            & 1.39            & 0.69            & 0.69            & 1.39            & 2.78 \\\hline
                    \end{tabular}
                    \caption{Análisis sensibilidad Ingresos}
                \end{table}
                
              \item {\bf Inversiones}\\
                \begin{table}[h!t]
                    \centering
                    \scriptsize
                    \begin{tabular}{|c|r|r|r|r|r|r|}
                        \hline
                        {\bf Inversión}      & -20\%           & -10\%           & -5\%            & +5\%            & +10\%           & +20\% \\\hline
                        Capital Trabajo      & \$4,990,400.00  & \$6,113,240.00  & \$5,926,100.00  & \$6,549,900.00  & \$6,861,800.00  & \$7,485,600.00 \\\hline
                        Capital Fijo         & \$4,800,000.00  & \$5,880,000.00  & \$5,700,000.00  & \$6,300,000.00  & \$6,600,000.00  & \$7,200,000.00 \\\hline
                        Capital Intangible   & \$1,200,000.00  & \$1,470,000.00  & \$1,425,000.00  & \$1,575,000.00  & \$1,650,000.00  & \$1,800,000.00 \\\hline
                        \hline
                        {\bf Flujos de caja} & -20\%           & -10\%           & -5\%            & +5\%            & +10\%           & +20\% \\\hline
                        Año 0                & -\$2,198,080.00 & -\$2,692,648.00 & -\$2,610,220.00 & -\$2,884,980.00 & -\$3,022,360.00 & -\$3,297,120.00\\\hline
                        Año 1                & -\$2,351,776.00 & -\$3,439,825.60 & -\$3,258,484.00 & -\$3,862,956.00 & -\$4,165,192.00 & -\$4,769,664.00\\\hline
                        Año 2                & -\$1,371,889.12 & -\$2,385,258.95 & -\$2,216,363.98 & -\$2,779,347.22 & -\$3,060,838.84 & -\$3,623,822.08\\\hline
                        Año 3                & \$9,166,143.41  & \$10,288,983.41 & \$10,101,843.41 & \$10,725,643.41 & \$11,037,543.41 & \$11,661,343.41\\\hline
                        \hline
                        {\bf VAN }           & \$746,440.20    & -\$722,225.54   & -\$477,447.92   & -\$1,293,373.33 & -\$1,701,336.04 & -\$2,517,261.45 \\\hline
                        {\bf Sensibilidad}   & 1.84            & 0.18            & 0.46            & 0.46            & 0.92            & 1.84 \\\hline
                    \end{tabular}
                    \caption{Análisis sensibilidad Inversiones}
                \end{table}

        \end{itemize}

       \begin{center}
       \begin{tikzpicture}
           \begin{axis}[
               xlabel=$\text{Porcentaje de variación ingresos}$,
               ylabel=$\text{VAN}$]
           \addplot[smooth,mark=*,blue] plot coordinates {
               (-20,-3346003.29)
               (-10,-2115706.96)
               (-5,-1500558.79)
               (0, 885410.62)
               (+5,-270262.46)
               (+10,344885.71)
               (+20,1575182.04)
           };
           \end{axis}
       \end{tikzpicture}
       \end{center}

       \begin{center}
       \begin{tikzpicture}
           \begin{axis}[
               xlabel=$\text{Porcentaje de variación inversiones}$,
               ylabel=$\text{VAN}$]
           \addplot[smooth,mark=*,blue] plot coordinates {
               (-20,746440.20)
               (-10,-722225.54)
               (-5,-477447.92)
               (0, 885410.62)
               (+5, -1293373.33)
               (+10,-1701336.04)
               (+20,-2517261.45)
           };
           \end{axis}
       \end{tikzpicture}
       \end{center}

\newpage
    \item {\bf Conclusiones, analizando detalladamente la información obtenida del flujo de caja y de los indicadores económico? (10p)}

          \red{Respuesta:}

          Del flujo de caja podemos decir, que el presente proyecto no es muy alentador en el sentido del tiempo
          en comenzar a ver cifras positivas, luego de cada año de desarrollo, pues como se señala anteriormente
          sólo en el tercer año tenemos un resultado positivo.

          Aparte de los problemas de liquidez que pueden notarse a primera vista,
          podemos también analizar la viabilidad del proyecto, las cuales
          lamentablemente según el calculo del VAN, nos indica la inviabilidad del proyecto.

          Considerando la sensibilización podemos decir que los ingresos son una variable
          mucho más importante en el proyecto en relación a las inversiones,
          lo cual tiene sentido, pues a cualquier proyecto le conviene tener más ingresos,
          que a poder realizar una inversión más grande al principio del proyecto.
          En ambos gráficos está reflejado el aumento y disminución del VAN en cada caso
          de sensibilización, lo que nos lleva a concluir que sólo con un aumento
          de los ingresos, podremos tener un VAN positivo, es decir, el proyecto
          podría ser viable.

          Observando los valores de los indicadores económico podemos señalar que el proyecto
          no es recomendable, debido a que el VAN es negativo, por lo que debiese ser rechazado,
          ya que la inversión produciría ganancias por debajo de la rentabilidad exigida.

          El PAYBACK no hace más que aclarar las respuestas que se pueden obtener del VAN,
          ya dentro del horizonte del proyecto, no podemos identificar un plazo de recuperación
          a la inversión realizada.

          Con el resultado del TIR, podemos darnos cuenta que de una tasa de 15\%,
          debemos bajarla por lo menos hasta un 9\% para poder poseer un proyecto viable,
          pero como bien sabemos, el TIR no representa muy bien la realidad,
          por lo que con el resultado del TIRM nos damos cuenta que necesitamos una disminución
          de 15\% a 12\% para que nuestro proyecto pueda ver la luz.

          En términos generales, creo que el proyecto es bastante poco alentador,
          sobre todo, pues he desarrollado una confianza al VAN, por los documentos leídos,
          así que a penas calculé el VAN, me di cuenta que el proyecto no daría buenos resultados.
           

\end{enumerate}
