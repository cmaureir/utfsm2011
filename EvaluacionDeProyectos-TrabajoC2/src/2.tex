\section{Segunda pregunta}

{\bf ¿En qué consiste la sensibilización y por qué es importante considerarla en la evaluación de un proyecto? (10p)}

\red{Respuesta:}


Es una subetapa dentro de la construccion del flujo de caja,
junto con la medicion de rentabilidad y el analisis de variables cualitativas.


identificar aquellos aspectos que pueden tener un comportamiento distinto al previsto
y que al ocurrir puedan determinar cambios importantes en la rentabilidad calculada.

las variables de sensibilizacion son propias de cada proyecto.


La sensibilización del proyecto: Se sensibiliza sólo aquellos aspectos que podrían, al tener mayores posibilidades de un comportamiento distinto al previsto, determinar cambios importantes en la rentabilidad calculada.


-----

21.- ANÁLISIS DE SENSIBILIDAD
Análisis de Sensibilidad

Se utiliza para determinar que tan sensible es una situación o un proyecto a las diversas variables, a fin de que se le asigne a cada una de ellas la importancia y consideración apropiadas.

Análisis comparativos en que se cambian los datos del análisis financiero para determinar los efectos sobre los indicadores financieros.
¿Cómo tratar la incertidumbre de datos?
¿Hasta qué punto son sensibles las medidas del proyecto ante cambios en los costos y beneficios estimados?
¿Cuál es la estabilidad del VAN, la TIR y la Relación B/C?
El análisis de sensibilidad se justifica, ya que muchos datos son estimaciones y/o promedios, en los proyectos de mediano y largo plazo, hay mucha incertidumbre con respeto a los rendimientos y precios de los productos finales y puede mostrar métodos para mejorar el diseño de los componentes de un proyecto.

Datos Típicos de un Análisis de Sensibilidad de un Proyecto

Los datos fundamentales o variables a las cuales generalmente se les realiza un análisis de sensibilidad son:

Relaciones técnicas (estimado): rendimientos, tasas de crecimiento o competitividad, estándares.
Precios (productos sobre tiempo)
Construcción y mantenimiento.
Duración/vida del Proyecto.
Tasas de descuento.
Etapas de un Análisis de Sensibilidad

Las etapas básicas de un análisis de sensibilidad son:

Determinar los rangos (sensibilidad estimada) de los costos y beneficios importantes para el Proyecto.
Hacer los análisis (Medidas de valor de proyectos) con datos diferentes.
Utilizar muchas combinaciones de escenarios diferentes y combinar los efectos de cambios biofísicos, sociales, políticos y económicos.
Evaluar los cambios, magnitud e importancia de los resultados en la clasificación y evaluación del proyecto.


----

En la construcción de un Proyecto de Mejoramiento Educativo se suceden una serie de etapas que procuran su arraigo institucional (pertenencia) así como la definición de objetivos adecuados y útiles en vistas de las prioridades de cada escuela y comunidad (pertinencia). Se procede de esta manera a desarrollar un ciclo de proyecto en el que pueden observarse una serie de etapas sucesivas: • Sensibilización, etapa en la cual es necesario animar y motivar a los actores escolares y de la comunidad, paa que se involucren en la iniciativa de generar cambios en la escuela


Formulación, toda la comunidad educativa se involucra en visualizar los logros y “resultados esperados” del proyecto a partir de una situación- problema, definición de objetivos y construcción de plan de acción. • Ejecución, consiste en la implantación y puesta en práctica de las actividades previstas en el proyecto • Seguimiento y monitoreo, proceso por el cual se hace un acompañamiento y verificación del desarrollo del proyecto y sus resultados, incluyendo ajustes al plan original • Evaluación, proceso que recoge aprendizajes y aporta información sobre el logro de metas y objetivos, sirviendo de insumo para reabrir el ciclo de proyecto


ciclo



-----


6.- Análisis de Sensibilidad

La medición de la rentabilidad analizada anteriormente sólo evalúa el resultado de uno de los escenarios proyectados, el cual es elegido por el analista con un criterio distinto (muchas veces) al que tendría el inversionista, por la aversión al riesgo de ambos y la perspectiva desde donde se analizan los problemas es diferente.

Es necesario, que al formular un proyecto se entregue los máximos antecedentes para que quien deba tomar la decisión de emprenderlo disponga de los elementos de juicio suficientes para ello.

Con este objetivo, y como una forma de agregar información a los resultados pronosticados del proyecto, se puede desarrollar un análisis de sensibilidad que permita medir cuán sensible es la evaluación realizada a variaciones en uno o más parámetros decisorios.

En este capítulo se presentan distintos modelos de sensibilización que se pueden aplicar directamente a las mediciones del valor actual neto, tasa interna de retorno y utilidad. Los modelos aquí presentados son de carácter económico, la sensibilización es aplicable al análisis de cualquier variable del proyecto, como la localización, el tamaño o la demanda.

6.6.4   Consideraciones preliminares

La importancia del análisis de sensibilidad se manifiesta en el hecho de que los valores de las variable que se han utilizado para llevar a cabo la evaluación del proyecto pueden tener desviaciones con efectos de consideración den la medición de resultados.

El análisis de sensibilidad, a través de los diferentes modelos que se definirán posteriormente, revela el efecto que tienen las variaciones sobre la rentabilidad en los pronósticos de las variables relevantes.

Visualizar qué variables tienen mayor efecto en el resultado frente a distintos grados de error en su estimación permite decidir acerca de la necesidad de realizar estudios más profundos de esas variables, para mejorar las estimaciones y reducir el grado de riesgo por error.

La repercusión que un error en una variable tiene sobre el resultado de la evaluación varía según el momento de la económica del proyecto en que ese error se cometa.

Dependiendo del número de variable que se sensibilicen en forma simultánea, el análisis puede clasificarse como unidimensional o multidimensional. En el análisis unidimensional, la sensibilización se aplica en una sola variable, mientras que en multidimensional se examinan los efectos sobre los resultados que se producen por la incorporación de variable simultánea en dos o más variables relevantes.

Aún cuando la sensibilización se aplica sobre las variables económicas financieras contenidas en el flujo de caja del proyecto, su ámbito de acción puede comprender cualquiera de las variables técnicas o mercado, que son en definitiva, las que configuran la proyección de los estados financieros.

6.6.5   El Modelo unidimensional de la sensibilización del VAN

El análisis unidimensional de la sensibilización del VAN determina hasta dónde puede modificarse el valor de una variable para que el proyecto siga siendo rentable.

Se define el VAN de equilibrio como cero por cuanto es el nivel mínimo de aprobación de un proyecto. De aquí que al hacer el VAN igual a cero se busca determinar el punto de quiebre o variabilidad máxima de una variable que resistiría el proyecto.

Como su nombre lo indica, y aquí radica la principal limitación del modelo, sólo se puede sensibilizar una variable por vez.

El principio fundamental define a cada elemento del flujo de cajo como el de más probable concurrencia.

Como se planteó en el capitulo 15, el VAN es la diferencia entre los flujos de ingresos y egresos actualizados del proyecto.

Para calcular la cantidad producida y vendida que hace al VAN igual a cero, deberá procederse de igual manera, observándose que la variable se encuentra tanta en la cuenta de ingresos como en la de costos variables. El mismo procedimiento se sigue para sensibilizar cualquier otra variable.

6.6.6   El modelo multidimensional de la sensibilización del VAN

El análisis multidimensional, a diferencia del unidimensional, además de incorporar el efecto combinado de dos o más variables, busca determinar de qué manera varía el VAN frente a cambios en los valores de esas variables como una forma de definir el efecto en los resultados de la evaluación de errores en la estimaciones.

El error la estimación se puede medir por la diferencia entre el valor estimado en la evaluación y otros que pudiera adoptar la variable eventualmente.

El modelo que se presenta a continuación considera flujos de caja constantes para simplificar la exposición, se trabajará con valores actuales y no con valores actuales netos, vales decir, excluyendo la inversión inicial, porque ésta para a ser irrelevante en la comparación al ser similar para ambas estimaciones, salvo que sea la variable por sensibilizar.

Para determinar el efecto potencial de los errores en los datos de entrada del modero del valor actual, se supondrá que la tasa de descuento permanecerá constante, sólo se trabajará con errores en la estimación de la vida útil, del flujo de caja o de ambos.

El análisis multidimensional adapta el unidimensional haciendo cero todas las variaciones, con excepción de las correspondientes a las variables por sensibilizar.

El proyecto será rentable si la diferencia entre el valor actual de las estimaciones es mayor o igual a la inversión inicial.

6.6.7    El modelo de sensibilidad de la TIR

Se definió la TIR como aquella tasa de descuento que hace igual a cero el VAN del flujo de caja del proyecto.

Para medir los efectos de los errores en las estimaciones se recurre al mismo procedimiento indicado para el análisis multidimensional del VAN.

La sensibilización de la TIR se efectúa calculando los errores EF (error porcentual) y ER (error porcentual en la duración del proyecto). los distintos valores de las variables, procediendo a determinar el valor de ER que haga la ecuación igual a cero.

Se analiza el efecto de una sola variable dejando las demás constantes, se puede apreciar que los errores en la estimación del flujo de caja se encuentran linealmente relacionados con errores en las tasas de rendimiento. No sucede así entre los errores en la vida útil y las tasas de rentabilidad.

El modelo aquí propuesto también pude aplicarse para investigar el efecto de errores combinados, es decir, cuando se producen cambios en más de una variable simultáneamente.

Aunque los flujos de caja positivos y negativos de igual valor absoluto inducen a errores positivos y negativos proporcionales en la tasa de rendimiento, no sucede lo mismo con errores en la duración, pues la tasa de rendimiento es más sensible a los errores negativos de duración que a los positivos.

Al mantener constante la magnitud de los errores de entrada al modelo, a medida que aumenta la tasa esperada de rendimiento, decrece la magnitud de los errores porcentuales inducidos en la tasa de rendimiento.

Esto supone que incertidumbre que rodea a los parámetros de presupuesto de capital en el caso de proyectos marginales puede ser mayor que en el caso de los proyectos que posean tasas de rentabilidad esperada mayor.

6.6.8   Usos y abusos de la sensibilidad

Básicamente, la sensibilización se realiza para evidenciar la marginalidad de un proyecto, para indicar su grado de riesgo o para incorporar valores no cuantificados.

Determinar la marginalidad de un proyecto es relevante, puesto que el monto del VAN calculado no representa una medida suficiente para calcular la proporcionalidad de los beneficios y costos del proyecto. El análisis de sensibilidad muestra cuán cerca de margen se encuentra en resultado del proyecto al permitir conocer si un cambio porcentual muy pequeño en la cantidad o precio de un insumo o del producto hace negativo el VAN calculado.

Al ser el flujo de caja, sobre el que se basa la evaluación, el resultado de innumerables estimaciones acerca del futuro, siempre será necesaria su sensibilización.

De aquí de desprende cómo se puede emplear este análisis para ilustrar lo riesgoso que puede ser un proyecto. Si se determina que el valor asignado a una variable es muy incierto, se precisa la sensibilización del proyecto a los valores probables de esa variable. Es muy sensible a esos cambios, el proyecto es riesgoso. El análisis de sensibilidad, en estos términos, es útil para decidirse a profundizar el estudio de una variable en particular o, a la inversa, para no profundizar más su estudio.

Aún incorporando variables cualitativas en la evaluación, preciso que éstas sean de alguna forma expresadas cuantitativamente. Esto mismo hace que el valor asignado tenga un carácter incierto, por lo que se requiere su sensibilización.




----




De acuerdo a Nassir Sapag (2000) los criterios de evaluación no miden la rentabilidad
del proyecto, sino que sólo miden la de uno de los tantos escenarios futuros posibles. Los
cambios que casi con certeza se producirán en el comportamiento de las variables del entorno,
harán que sea prácticamente imposible esperar que la rentabilidad calculada sea la que
efectivamente tenga el proyecto implementado. Pero, como también decíamos, más vale una
buena aproximación que no tener esta información para apoyar la toma de la decisión de
emprender el proyecto.
Frente a esta limitación y a la necesidad de entregar al inversionista el máximo de
información, surgen los modelos de sensibilidad como un complemento de toda evaluación.
El método más tradicional y común es el que analiza qué pasa con VPN cuando se
modifica el valor de alguna variable que se considera susceptible de cambiar durante el
período de evaluación.
El modelo propone que se confeccionen tantos flujos de caja como posibles
combinaciones que se identifiquen entre las variables.
El modelo unidimensional permite trabajar con una sola variable cada vez. De ahí el
nombre de unidimensional.
Analizar qué pasa con el VPN cuando se modifica el valor de una variable estimada en
el flujo inicial para que el proyecto siga siendo atractivo para el inversionista.


Este método es mucho más eficiente, por cuanto un solo valor límite, el cual, al ser
conocido por el inversionista, le permite incorporar a la decisión su propia aversión al riesgo.
Con este método se busca el punto límite; o sea, determinar hasta dónde podría bajar la
demanda para que el proyecto siga siendo conveniente. Esto es lo mismo que buscar la
cantidad que hace el VPN igual a cero.

