\section{Segunda pregunta}

{\bf ¿En qué consiste la sensibilización y por qué es importante considerarla en la evaluación de un proyecto? (10p)}

\red{Respuesta:}

Podemos entender la sensibilización simplemente como la verificación de algunos aspectos que podrían provocar
un compartimiento imprevisto en nuestro proyecto, ayudándonos así a determinar algunos cambios importantes en la rentabilidad.

Tenemos que considerar que todos los cálculos y conclusiones que obtengamos el proceso de la evaluación de un proyecto
son para un escenario determinado fijo, por lo que se aleja mucho de lo que realmente ocurre,
es por esto que la sensibilización es vital a la hora de trabajar con la rentabilidad.

Como ya señalamos, las variables en el proceso de evaluación de proyectos pueden sufrir de desviaciones,
que afecte considerablemente la medición de los resultados, por lo tanto en ésta etapa se podrá
revelar dicho efecto de algunos pronósticos de las variables más relevantes del proyecto.

Al determinar dichas variables importantes, podremos mejorar
nuestras estimaciones y así reducir el grado de riesgo por el error
de nuestro análisis.

Dependiendo de cuantas variables se van a sensibilizar al mismo tiempo,
este se clasifica en unidimensional y multidimensional,
considerando sólo 1 variable o más de una variable en cada caso.

Considerando que la sensibilización se aplicará sólo sobre variables económicas financieras
que son parte del flujo de caja del proyecto, el ámbito en el cual actúan puede
implicar cualquiera de las otras variables, ya sean del mercado, o técnicas,
que puedan involucrarse en la proyección de los estados financieros.

El modelo utilizado en el presente trabajo, fue el unidimensional,
considerando como mecanismo de medida el VAN, que utilizando la variación
porcentual, podemos ver el efecto sobre nuestro proyecto.

Al ver las variables críticas del proyecto, lo que estamos haciendo
es poder determinar las marginalidad del proyecto,
ya sea para ver el riesgo o evaluar la incorporación de valores
no cuantificados.

Considerando la marginalidad, la importancia que posee se debe a que el monto del VAN que estamos
calculando, no representa un valor suficiente para poder calcular la proporcionalidad de todos los costos
del proyectos o los mismos beneficios.

Como basamos nuestra evaluación en el flujo de caja, el resultado
será de todas las estimaciones de estados futuros,
por lo que siempre será vital para un proyecto sensibilizar algunas variables.
