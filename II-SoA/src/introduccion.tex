En el campo de la astrofísica,
han surgido distintos problemas que debido a la complejidad
o intensidad del cálculo, han sido el objetivo de muchos
científicos e investigadores a la hora de realizar
aplicaciones en sistemas computacionales de alto desempeño.

Uno de dichos problemas se basa en los principios
fundamentales del movimiento particular, relacionándolo con
la fuerza gravitacional, nos referimos al problema de \emph{n-cuerpos}.

% definicion del problema
El problema de los \emph{n-cuerpos} es el problema de poder
predecir el movimiento de un conjunto determinado de cuerpos celestes,
los cuales interactuarán unos con otros debido a la fuerza
gravitacional.

Cada objeto celeste tendrá una determinada \emph{masa} $(m)$
y una determinada ubicación en el espacio
tridimensional $(x,y,z)$, además de tener una velocidad inicial
determinada en cada dirección $(v_{x},v_{y},v_{z})$ y finalmente
tendrá una cierta aceleración inicial de cero $(a_{x},a_{y},a_{z})$.
Cabe señalar que tanto la \emph{posición}, la \emph{velocidad} y la \emph{aceleración},
serán actualizadas dependiendo de la interacción con otros cuerpos.

% ejemplo informal
Un ejemplo clásico de lo que es el problema de \emph{n-cuerpos},
es el movimiento de los planetas en nuestro sistema solar,
el cual se vé afectado tanto de las propiedades del Sol,
como de todos los otros cuerpos que se encuntran entre sus órbitas.

% interes del problema
El factor de interés del problema de \emph{n-cuerpos}
es que los algoritmos numéricos que se desarrollan para resolver
la dinámica de este problema, puede aplicarse no sólo en el área de la astrofísica,
con campos como \emph{Celestial mechanics}, \emph{Dense stellar systems},
\emph{Sphere of influence of a massive BH}, y \emph{Galaxy dynamics and cosmology},
si no que también es ampliamente utilizado en el área de dinámica de fluídos,
el cual por  ejemplo puede verse reflejado en el trabajo de Gingold et al.~\cite{gingold}
donde se toma como objetivo principal un método de resolución de modelos físicos,
mas que la interacción misma de las particulas.

Otra área importante dentro de la dinámica de fluidos son los \emph{Vortex Methods},
los cuales son una técnica para realizar la simulación de distintos flujos turbulentos
de un fluído en particular. Sin embargo, las aplicaciones no se quedan sólo en la teoría
pues por ejemplo, se han realizado simulaciones de humo en tiempo real
para utilizarlo en el desarrollo de video juegos~\cite{gourlay}.

% 	¿Por qué es dificil?
% 		¿Es NP-completo?
% 		Debilidades de lo existente
% 		Costo versus calidad de las soluciones

La resolución del problema de los \emph{n-cuerpos} consta de tres
aspectos claves, que determinarán la dificultad del cálculo.

Primero se requiere un escenario inicial para dicho problema,
por lo cual existen distintas alternativas, las cuales pueden ser
generar las posiciones aleatoreamente, que de por sí conlleva un problema,
o ocupar modelos existentes para generar una situación inicial,
como \emph{Plummer model} \red{CITE}.

Otro aspecto importante es que como debemos trabajar con la interacción
entre los cuerpos, se debe utilizar un método para integrar las ecuaciones
del movimiento, para así poder actualizar tanto \emph{posición}, \emph{velocidad}
y \emph{aceleración}. 

En este punto es donde se han utilizado variados métodos ampliamente conocidos,
como por ejemplo \emph{Forward and Backward Euler}, los cuales no son recomendados
debido a su poca precisión. Desde ese punto que otro métodos han ido mejorando
el funcionamiento de las ecuaciones de movimiento, como \emph{Leapfrog integration}.
Finalmente los métodos más recomendados son \emph{Runge Kutta} y \emph{Two step Adams-Bashworth},
los cuales poseen una precisión mucho mejor que los métodos anteriores, pero son
más caros a nivel de cálculo.

Finalmente, el punto más importante para el presente problema es el cálculo
de la fuerza entre las particulas, tema principal del presente trabajo,
ya que si consideramos que la fuerza ejercida sobre una particula está determinada
por todo el resto de particulas en un sistema determinado, el orden del algoritmo
crecerá en forma cuadrática al aumentar la cantidad de particulas,
tema por el cual variados investigadores han realizado técnicas para disminuir
el orden de $O(n^2)$ que posee en su versión inicial, ocupando un enfoque
\emph{Particle-Particle}.
