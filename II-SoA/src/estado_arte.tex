% Criticar soluciones existentes, no agresivamente.
% 	Analisis -> Criterio (que usaremos para analizar nuestro propio trabajo)
% 	Definir criterios de evaluación.
% 	Explicar lo que implica y por que son importantes.

% Introduccion del estado del arte explicando enfoque

% formula de calculo de fuerza

El cálculo de la fuerza en su forma general está dado por,
$$f_{ij} =G \cdot \frac{m_i \cdot m_j}{||r_{ij}||^{2}} \cdot \frac{r_{ij}}{||r_ij||}$$,
donde las posiciones iniciales son $x_i$,
las velocidaddes son $v_i$,
teniendo a $i$, entre los valores, $1 \leq i \leq N$,
la masa de los cuerpos $i$ y $j$ determinada por $m_i$ y $m_j$,
siendo $r_{ij} = (x_j - x_i )$ vector de distancia entre los cuerpos $i$ y $j$
y finalmente $G$, constante gravitacional. ($6,67428 \times 10^{-11} m^{3} kg^{-1} s^{-2}$)

% Explicación de métodos para calculo fuerza

\subsection{Particle-Particle}
% n2

\subsection{Particle-Mesh}
% n + ng log ng

\subsection{Treecodes}
% n log n

\subsection{Multipole methods}
% n

