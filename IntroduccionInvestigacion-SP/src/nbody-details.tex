% aproximaciones para resolver el problema

%\subsection{Particle-Particle (PP)}

%This approach is the simplest way to address
The Particle-Particle (PP) method is the simplest way to address
the task of calculating the exerted force by all the bodies on a single body.

Broadly, the method consists of:
\begin {itemize}
	\item Collect forces on a given particle (using the previous indicated formula).
	\item Integrate the motion equations.
	\item Update the counter.
	\item Repeat the process.
\end {itemize}

Note that the process in which the equations are solved,
consists of two first order differential equations,
to calculate the acceleration and speed,
and also uses an integration method for new positions and
velocities of the bodies.

As this the simplest scenario,
we realize that the process is of order $O(n^{2})$,
so for any algorithm, not a desirable scenario.

Since this initial solution, completely theoretical,
without any improvement, any text that refers integration methods will be useful,
but is also convenient to use simulation references,
as is the work of Gould and Tabochnik~\cite{methods}.

%%%

This work will consider only an implementation
of the PP method, but is important to clarify
that there was other approximations more efficient
like the Particle-Method (PM) method,
which present a to $O(n + ng\log ng)$ order,
(ng: amount of the used vertex);
with $O(n\log n)$ algorithm order.
Other algorithm, is the model proposed by Barnes \& Hut~\cite{treecode} in 1986,
called Tree Code, offering a $O(n\log n)$ algorithm for the force
calculation.
To the point reached by the TC algorithms,
which have been studied to obtain a good performance
on GPU clusters, making it a typical test case
in order to obtain excellent performances,
allowing to have been winners of several awards,
as the case Hamada et al.~\cite{hamada},
which was awarded the Gordon Bell prize~\footnote{\url{http://awards.acm.org/bell/}}
several times, obtaining in one of his last work
190 TFlops using TC algorithms to solve the \emph{n-body} problem.
Finally, the \emph{Fast Multipole Method} is one of the methods
mostly used, for its low complexity and provides high accuracy,
being proposed by Leslie Greengard in his doctoral thesis~\cite{leslie},
this algorithm is known to propose an order usually of $O(n)$.

%%%


%\subsection{Particle-Mesh (PM)}
%
%This method generates a mesh in all the space where we have bodies,
%thus seeking to obtain the approximate force at different points in the mesh.
%
%In addition, the last method differential operators are replaced by
%finite difference approximations, reducing the calculation,
%since the force and potential of the positions of the bodies
%are obtained by interpolation of the array of
%the values of the mesh.
%
%Another important feature of the mesh, is that it defines a kind
%of density, which are calculated by the exerted loads of every body
%to a determinate mesh point.
%
%Broadly, the method consists of:
%\begin{itemize}
%    \item Assign the load to the mesh, taking into account the relationship between
%          body mass and mesh density.
%    \begin{itemize}
%        \item The best way is when bodies are varied
%              nearby, to reduce force fluctuations.
%    \end{itemize}
%    \item Solve the force equation field on the mesh,
%          which can be solved using Poisson's equation: $\delta^{2}(\phi) = 4\cdot\pi\cdot G\cdot \rho$.
%    \item Calculate the force field from the potential of the mesh.
%    \item Interpolate the force on the mesh to determine the forces
%          on their bodies.
%    \item Integrate forces to obtain the new bodies positions and velocities.
%    \item Update counter.
%    \item Repeat the process.
%\end{itemize}
%
%This method presents a more quick calculation, in comparison to the \emph{Particle-Particle} method,
%decreasing of the same way the algorithm order, to $O(n + ng\log ng)$,
%where $ng$ is a amount of mesh vertex.
%
%A key problem, is the complexity caused to solve a Poisson equation,
%besides not being a recommended model in cases in which we like to study
%the bodies collision, because we approximate groups of bodies in
%some mesh vertex.
%
%Because the PM model is a model that takes quite a while as a available
%tool to scientists and researchers, has plenty of room to give rise to new research,
%such as the work of Villumsen~\cite{Villumsen}, who proposes
%a new PM scheme based on hierarchy, which differs from the initial scheme,
%since the Poisson equation is solved using FFT methods,
%in cubic mesh, considering a tree structure,
%which is clearly based its structure TC algorithms.
%
%On the other hand, there have been computationally more realistic models,
%such as working Pen~\cite{pen}, where presents a PM model, 
%but with the difference that it uses a dynamic coordinate system
%which is dynamic auto-adapted according to the distribution of bodies density,
%becoming an early rapid resolution in the field of n-body algorithms.
%
%
%\subsection{Treecodes (TC)}
%
%This model was proposed by Barnes \& Hut~\cite{treecode} in 1986,
%and today remains one of the most widely used base models,
%because from this mechanism, many researchers have made minor
%modifications to obtain either a lower or less run time order
%of the algorithm.
%
%The method is based on the simple idea that when you have a certain
%number of bodies far enough, you can approach them all into one single body,
%determining its position from the center of mass, obtaining a great body,
%instead of several dispersed.
%
%Besides being a brilliant scheme to group the distant bodies,
%is divided recursively all the bodies in a structure called \emph{quad-tree}
%in two dimensions, and \emph{oct-tree} in three dimensions, which are similarly
%represented as a tree, inserting the idea that the root of the tree represents
%the entire space, and adding childrenis, these represent divisions
%within the space where they are.
%Finally we can say that every external node
%will represent a single body, while the internal,
%the group of bodies representing approximate.
%
%The procedure for calculating the force on a body is:
%\begin{itemize}
%    \item We travel all nodes in the tree,
%    \begin{itemize}
%        \item If is an external node, we use the same idea of the \emph{Particle-Particle} approach.
%        \item If is an internal node,
%        \begin{itemize}
%            \item If is far enough, we use the approach.
%                  To determine if the body is far enough,
%                  it used a ratio between the region width where is the
%                  node and the distance to the center of mass from this body.
%                  This ratio compares with a threshold that will give us the answer.
%            \item If is not far enough, we enter the sub-trees.
%            \item Repeat the process
%        \end{itemize}
%    \end{itemize}
%\end{itemize}
%
%The procedure to construct the tree, i.e., enter a new $b$ body in the tree,
%rooted in a $x$ node, consists of:
%\begin{itemize}
%    \item If $x$ does not have a body, $b$ is entered in that position.
%    \item If $x$ is an internal node, we update the center of mass and total mass.
%          (Recursion, until it is not an internal node).
%    \item If $x$ is an external node, and contains a $c$ node,
%          as there would be two bodies in the same region,
%          we need to subdivide recursively until both bodies
%          will be located in different regions.
%\end{itemize}
%
%The main plus point,
%is that the forces on all bodies are obtained by operations
%with $O(n\log n)$ order.
%
%The problem is that you lose accuracy approximating the bodies (far enough)
%into a single one.
%
%After the first approach of TC algorithms,
%some researchers began to improve certain aspects
%in which efficiency was not very favorable,
%as in the work of Jernigan et al.~\cite{jernigan},
%which is the main approach to handle the
%generated error by TC algorithms,
%setting limits to keep the energy conservation law
%in the bodies.
%
%To the point reached by the TC algorithms,
%which have been studied to obtain a good performance
%on GPU clusters, making it a typical test case
%in order to obtain excellent performances,
%allowing to have been winners of several awards,
%as the case Hamada et al.~\cite{hamada},
%which was awarded the Gordon Bell prize~\footnote{\url{http://awards.acm.org/bell/}}
%several times, obtaining in one of his last work
%190 TFlops using TC algorithms to solve the \emph{n-body} problem.
%
%\subsection{Multipole methods}
%
%The \emph{Fast Multipole Method} is one of the methods
%mostly used, for its low complexity and provides high accuracy,
%being proposed by Leslie Greengard in his doctoral thesis~\cite{leslie}.
%
%Broadly speaking, is a \emph{Treecode} which use
%two representations of the potential field,
%which are a far field, it would be a \emph{multipole}
%and local expansions.
%
%Since this method uses a quick calculation of the potential field,
%it is very easy to make the computationally procedure,
%compared to the force calculation, being this last one
%a vector which is contrary to the scalar which is the potential $(\phi (x, y, z))$.
%Here it is necessary to remember a basic physical equivalence,
%where the force can be represented as the negative potential gradient.
%
%$$F = - \nabla P$$
%
%The method strategy is to calculate an small expression
%for the potential, using multipole expansions, which
%resemble a Taylor expansion,
%with high accuracy when its get bigger values
%from the expression $x^{2}+y^{2}+z^{2}$.
%
%As mentioned above,
%this method shares the idea of tree,
%from the \emph{Treecode} method
%but differs in the following aspects:
%
%\begin{itemize}
%    \item This method calculates the potential at all points,
%          not force, as does \emph{Treecode}.
%    \item This method uses more information than the mass
%          and the center of the particles in a region, as being
%          a more precisely expansion, their complexity grows.
%    \item The decision whether a region (bodies set) is used or not
%          as one body, depends only on the position and region size,
%          not of the center of mass position of the region.
%\end{itemize}
%
%Being a more accurate method has its cost,
%which was demonstrated by McMillan et al.~\cite{mcmillan},
%which mention is not a good model for studying system collisions because
%it is slower compared to other methods and that has a constant time
%when evaluating a number of particles force, with an order $O(1)$.
%
%However, this model is widely used for applications in which the bodies
%have the same or similar time steps, when evaluating the motion equations.
%
%Although,
%this algorithm is known to propose an order usually $O(n)$,
%there is a work of Srinivas Aluru~\cite{srinivas},
%which ensures that the algorithm does not have that order,
%because is obtained only in the best case,
%neglecting the analysis of worst-case scenario for the algorithm.
%By analyzing each step of the algorithm,
%since the construction of the tree,
%where some constants defined by Greengard,
%which Srinivas justify which is not possible,
%concluding that the algorithm will behave
%at least an order of $O(N \log^{2} N) $.
