% explicación general del algoritmo y las diferencias
% de ambas implementaciones

The used algorithm is the more simplest
way to implement a \emph{Particle-Particle}
approach to the \emph{n-body} problem,
i.e. we will have a main loop, with a certain
time-steps, and inside we will update the
particles attributes, like the ``position'',
``acceleration'' and ``velocities''.

The main characteristics,
is that is a fine-grained algorithm,
it means, perform a lot of small
mathematical calculation, several times,
instead of a big task.

In pseudo-code, the algorithm idea will be
as follows:

\begin{center}
\begin{algorithmic}
\STATE $updateAccelerations()$
\FOR {$counter = 0 \to iterations$}
    \STATE $updatePositions()$
    \STATE $updateAccelerations()$
    \STATE $updateVelocities()$
\ENDFOR 
\end{algorithmic}
\end{center}

After a serial implementation profiling,
we can note that almost all the execution time ($98\%$)
is in the \emph{updateAccelerations()} function,
because inside we have a line which use a \texttt{math.h}
function called \emph{sqrt()}.

So the main goal will be optimize that function,
to optimize the execution time of our \emph{n-body} problem
implementation.

\subsection{OpenMP Implementation}
 
The main idea behind the OpenMP implementation
is to use the most important characteristic of this approach,
the simplicity.

When we use OpenMP, we are looking for a quick
optimization to our code, in this example,
after the profiling, we note that all the execution
time was in the \texttt{updateAcceleration()} function,
so we need to optimize that code portion.

So, the simplest way to optimize a \emph{for} loop,
will be to add a \texttt{\#pragma omp parallel for},
and works, but in this case, we have a nested \emph{for} loop,
inside another loop, so the previous pragma will not work.

One of the OpenMP characteristics to solve this issue,
is to privatize variables, that can be done in different ways,
for example:

\begin{itemize}
    \item \texttt{private(i)}:
            A \emph{private} variable means that the value is not dependent on any other thread,
            so each thread will have a copy of the variable.
            The internal behaviour is:
            \begin{itemize}
                \item Create new object of the same type in each thread,
                \item The references to the original object are replaces by the new one,
                \item We assume that the variable are uninitialized,
            \end{itemize}    
    
    \item \texttt{firstprivate(i)}:
            A \emph{firstprivate} variable is very similar to the \emph{private} variable,
            but provides an automatic initialization of the variables, according to the
            original value, before enter in the parallel section.

    \item \texttt{lastprivate(i)}:
            A \emph{lastprivate} variable is very similar to the \emph{firstprivate} but
            the value of the variable is from the last loop iteration.
            So the last value is copied back into the original variable.
\end{itemize}

In this case, we use the \texttt{private} OpenMP variable to the second
counter ($j$), because we need to each $i$-thread has a self $j$ counter
to calculate the force on a single body.

Finally, the code was programmed as follows:

\begin{lstlisting}[style=C]
void updateAccelerations(){
    int i,j;
    for (i = 0; i < n; i++) {
        bodies[i].ax = bodies[i].ay = bodies[i].az = 0;
    }
    #pragma omp parallel for private(j)
    for (i = 0; i < n; i++) {
        for(j=0; j<n; j++){
            if(j != i){
                accelerationCalc(i,j);
            }
        }
    }
}
\end{lstlisting}

You can note, that the only modification is only the
pragma line, and internally, OpenMP will distribute and
synchronize all the threads in the CPU's of the computers,
which is good is we are looking for quick programming
optimization, but not enough good if we like to
handle the synchronization by our-self, avoiding all the
possible overhead of the threads call.

We do not use critical sections,
because we do not have shared variables
in each iteration, or loop dependencies.

\subsection{Pthreads Implementation}

Using POSIX threads, we have more low level control
to the threads itself, instead the OpenMP approach.

In this case, we write the code thinking in obtain
the best results, using the CPU in a very efficient way.
This can be possible, because we perform an operation
between the amount of bodies and the number of cores,
so we distribute equally in each core a set of bodies,
in other words, we are change the algorithm grain,
from a ``fine'' to ``coarse'', grouping bodies
calculations.

As the calculation of the force on every body
depends of all the others bodies, we do not have
a calculation dependence, because we modify only the acceleration
using the position and mass of the other bodies,
so is not necessary to use \texttt{mutex} to avoid
data corruption or errors.

\begin{lstlisting}[style=C]
void updateAccelerations(){
    int i;

    for (i = 0; i < n; i++) {
        bodies[i].ax = bodies[i].ay = bodies[i].az = 0;
    }
    int bound = n/cores;
    for (i = 0; i < cores; i++) {
        threads[i].ini = i * bound;
        threads[i].end = threads[i].ini + bound;
        pthread_create(&threads[i].thread, NULL , accelerationCalc, (void *)&threads[i]);
    }

    for(int j=0 ; j < cores ; ++j){
        pthread_join(threads[j].thread, NULL);
    }

}
\end{lstlisting}

The previous Pthreads implementation follows three simple steps:

\begin{itemize}
    \item Calculate the bodies bound (per each core),
    \item Execute the \emph{accelerationsCalc()} function, per each thread (considering an amount of bodies).
    \item Join all the threads, and wait for his end.
\end{itemize}

Finally, we obtain a little more complicated approach to the parallelism,
having more control to the threads synchronization and execution,
but we are taking advantage of OpenMP, because the change from
\emph{fine-grained} to \emph{coarse-grained} because is more appropriate
to a CPU parallelization approach, to use the most resources that
we can obtain for each core.

\subsection{CUDA Implementation}

The GPU based implementation, using CUDA,
can be more complicated because is in essence C++,
but with some extra statements to handle the
differentiation between \emph{Device} and \emph{Host} functions,
which execute into the GPU and in the CPU respectively.

There are several research arround this direct method implementation
using CUDA, like the work of Belleman et al.~\cite{cuda1},
obtained good results, also, people from Nvidia
use this problem as a toy algorithm to try the power of GPU,
an example can be found in \cite{cuda2}.

The main idea of the implementation is the same,
but the only difference is that we have 240 cores (1.3 GHz),
instead of the 16 cores in the CPU (2.27 Ghz),
which give us a great scenario for a fine-grained algorithm.

The main kernel call is very simple,
besides the number of \emph{blocks per grid}
and the number of \emph{threads per block},
we need the main bodies array (\texttt{bodies}), the iteration variable (\texttt{iter}),
the time step in each iteration (\texttt{dt}) and the total of bodies (\texttt{N}). 

\begin{lstlisting}[style=C]
 nbody<<< blockNum, threadsPerBlock >>> (bodies,iter,dt,N);
\end{lstlisting}

The kernel implementation,
is based on the idea of separate each function
in a different \emph{device function},
so we have the need to synchronize all the used
threads after each device function call.

The following code, represent the CUDA implementation
based on the main pseudo-code idea.

\begin{lstlisting}[style=C]
__global__ void nbody(particle *bodies,float iter, float dt, int N){
    reset_accelerations(bodies,N);
    __syncthreads();
    update_accelerations(bodies, N);
    __syncthreads();
    
    for (float t = 0.0; t < iter; t += dt){
    
        update_positions(bodies,dt,N);
        __syncthreads();

        reset_accelerations(bodies,N);
        __syncthreads();

        update_accelerations(bodies,N);
        __syncthreads();
    
        update_velocities(bodies,dt,N);
        __syncthreads();
    }
\end{lstlisting}

We take advantage with this implementation,
because the GPU is physically designed
to work with several small calculation
in all the cores, base idea of the graphical
computation.

We use the idea of, \emph{one-thread}
means \emph{one-body}, which is different
to the previous implementations.
