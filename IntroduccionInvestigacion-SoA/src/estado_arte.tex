% Criticar soluciones existentes, no agresivamente.
% 	Analisis -> Criterio (que usaremos para analizar nuestro propio trabajo)
% 	Definir criterios de evaluación.
% 	Explicar lo que implica y por que son importantes.

% Introduccion del estado del arte explicando enfoque

% formula de calculo de fuerza

El cálculo de la fuerza en su forma general está dado por,
$$f_{ij} =G \cdot \frac{m_i \cdot m_j}{||r_{ij}||^{2}} \cdot \frac{r_{ij}}{||r_ij||}$$,
donde las posiciones iniciales son $x_i$,
las velocidades son $v_i$,
teniendo a $i$, entre los valores, $1 \leq i \leq N$,
la masa de los cuerpos $i$ y $j$ determinada por $m_i$ y $m_j$,
siendo $r_{ij} = (x_j - x_i )$ vector de distancia entre los cuerpos $i$ y $j$
y finalmente $G$, constante gravitacional. ($6,67428 \times 10^{-11} m^{3} kg^{-1} s^{-2}$)

% Explicación de métodos para calculo fuerza

\subsection{Particle-Particle (PP)}

El presente enfoque es la forma más simple de poder abordar
la tarea del calculo de la fuerza que ejercen los cuerpos sobre otros.

A grandes rasgos, el método consiste en:
\begin{itemize}
	\item Acumular las fuerzas sobre una partícula determinada (utilizando fórmula anteriormente señalada).
	\item Integrar las ecuaciones de movimiento.
	\item Actualizar el contador.
	\item Repetir el proceso.
\end{itemize}

Cabe señalar que el proceso en el cual se resuelven las ecuaciones,
consta de dos ecuaciones diferenciales de primer orden,
para poder calcular la aceleración y la velocidad,
y además se utiliza un método de integración, para obtener
las nuevas posiciones y velocidades de los cuerpos.

Como este panorama es el más sencillo,
podemos darnos cuenta de que el proceso es de orden $O(n^{2})$,
lo que para cualquier algoritmo, no es un escenario
deseable.

Al ser esta la solución inicial, completamente teórica,
sin ningún tipo de mejora, cualquier texto que haga
referencia a métodos de integración será útil,
pero también es conveniente recurrir a referencias
de simulación, como lo es el trabajo de Gould y Tabochnik~\red{methods}.

\subsection{Particle-Mesh (PM)}

Este método genera una malla en todo el espacio donde tenemos los cuerpos,
buscando así poder obtener la fuerza aproximándola en distintos
puntos de la malla.

Además, los operadores diferenciales del método pasado son reemplazados por
aproximaciones de diferencia finita, reduciendo el cálculo efectuado,
ya que la fuerza y los potenciales de las posiciones de los cuerpos
son obtenidos realizando una interpolación del arreglo de
los valores de la malla.

Otra característica importante de la malla, es que se define una especie
de densidad, las cuales son calculadas por las cargas que ejercen
cada cuerpo a un punto determinado de la malla.

A grandes rasgos, el método consiste en:
\begin{itemize}
	\item Asignar la carga a la malla, teniendo en cuenta la relación entre
		la masa del cuerpo y la densidad de la malla.
		\begin{itemize}
			\item La mejor forma será cuando variados cuerpos se encuentran
				cerca, para reducir las fluctuaciones de la fuerza.
		\end{itemize}
	\item Resolver la ecuación de potencia de campo sobre la malla,
		la cual puede ser resuelta utilizando la ecuación de Poisson: $\delta^{2} (\phi) = 4\cdot \pi \cdot G \cdot \rho$.
	\item Calculas la fuerza del campo a partir de los potenciales de la malla.
	\item Interpolar la fuerza sobre la malla para determinar las fuerzas
		sobre los cuerpos.
	\item Integrar las fuerzas para obtener las posiciones y velocidades
		de los cuerpos.
	\item Actualizar contador.
	\item Repetir el proceso.
\end{itemize}

Este método presenta un cálculo más rápido que el modelo \emph{Particle-Particle},
disminuyendo de la misma forma el orden del algoritmo a, $O(n + ng \log ng)$,
donde $ng$ es el número de vértices de la malla.

Un problema claro, es la complejidad que conlleva poder solucionar la ecuación de Poisson,
además de no ser un modelo recomendado para casos en que se quiera estudiar la colisión
entre cuerpos, ya que los trata casi como un sólo cuerpo, pues
aproxima cuerpos en los vértices de la malla.

Debido a que el modelo PM es un modelo que lleva bastante
tiempo como una herramienta disponible para los científicos e investigadores,
ha tenido gran cabida para dar pie a nuevas investigaciones, como
es el caso del trabajo de Villumsen~\cite{villumsen}, quien propone
un nuevo esquema de PM basado en jerarquía, que se diferencia del esquema inicial,
ya que la ecuación de Poisson es resuelta utilizando métodos de FFT,
en mallas cúbicas, considerando una estructura de árbol,
la cual es claramente basaba su estructura en los algoritmos TC.
Por otro lado, se han realizado modelos más realistas computacionalmente,
como es el caso de trabajo de Pen~\cite{pen}, donde plantea un modelo PM,
pero con la diferencia de que utiliza un sistema de coordenadas dinámico
la cual se va auto-adaptando de acuerdo a la densidad de la distribución
de los cuerpos, convirtiéndose en una temprana resolución rápida
en el campo de los algoritmos de n-cuerpos.

\subsection{Treecodes (TC)}

El presente modelo fue propuesto por Barnes \& Hut~\cite{treecode}
en el año 1986, y hoy en día sigue siendo uno de los modelos base más utilizado,
pues a partir de este mecanismo,
muchos investigadores han realizado pequeñas modificaciones,
para poder obtener, ya sea un menor tiempo de ejecución o menor orden
del algoritmo.

El método se basa en la simple idea de que al momento de tener una cierta
cantidad de cuerpos bastante lejos, es posible aproximarlos todos
a un sólo cuerpo, determinando su posición a partir del centro de masa,
obteniendo un gran cuerpo, en vez de varios dispersos.

Además de ser un esquema inteligente para poder agrupar los cuerpos
lejanos, va dividiendo recursivamente todos los cuerpos,
en una estructura llamada \emph{quad-tree} en dos dimensiones,
y \emph{oct-tree} en tres dimensiones, los cuales análogamente
tienen una representación como un árbol, insertando la idea
de que la raíz del árbol representa el espacio completo,
y a medida que van agregándose hijos, estos representarán
divisiones dentro del espacio donde se encuentran.
Finalmente podremos decir que cada nodo externo,
representará un único cuerpo, mientras que los internos,
representarán el grupo de cuerpos aproximado.

El procedimiento para calcular la fuerza sobre un cuerpo consiste en:
\begin{itemize}
	\item Recorremos todos los nodos del árbol,
	\begin{itemize}
		\item Si es un nodo externo,
			utilizamos la misma idea del enfoque \emph{Particle-Particle},
		\item Si es un nodo interno,
		\begin{itemize}
			\item Si está lo suficientemente lejos, utilizamos la aproximación.
				Para determinar si el cuerpo está suficientemente lejos,
				se utiliza un cociente entre el ancho de la región donde está
				el nodo y la distancia del presente cuerpo al centro de masa.
				Dicho cociente se compara con un umbral que nos entregará la respuesta.
			\item Si no está lo suficientemente lejos, nos adentramos en los sub-árboles.
		\item Repetimos el proceso
		\end{itemize}
	\end{itemize}
\end{itemize}

El procedimiento para construir el árbol,
es decir, ingresar un nuevo cuerpo $b$ en el árbol
con raíz en un nodo $x$, consta en:
\begin{itemize}
	\item Si $x$ no tiene un cuerpo, $b$ se ingresa en dicha posición.
	\item Si $x$ es un nodo interno, actualizamos el centro de masa y la masa total.
		(Recursión, hasta que no sea un nodo interno).
	\item Si $x$ es un nodo externo, y contiene un nodo $c$,
		como habrían dos cuerpos en la misma región,
		se va subdividiendo recursivamente hasta que ambos cuerpos
		queden en regiones distintas.
\end{itemize}

El principal punto a favor,
es que las fuerzas sobre todos los cuerpos se obtienen con operaciones
con orden $O(n log n)$.

El problema es que se pierde precisión al aproximar los cuerpos
en uno sólo al estar lo suficientemente lejos.

Luego del primer acercamiento de algoritmos TC,
algunos investigadores comenzaron a mejorar ciertos aspectos
en los cuales la eficiencia no era muy favorable,
como es el caso del trabajo de Jernigan et al.~\cite{jernigan},
el cual tiene como enfoque principal poder controlar
el error generado por los algoritmos TC,
estableciendo límites para respetar la conservación
de la energía de los cuerpos.

A tal punto han llegado los algoritmos TC,
que han sido objeto de estudio para poder obtener
un buen desempeño en clusters de GPU,
siendo así un caso típico de prueba
para poder obtener desempeños que han sido
ganadores de distintos premios, como es el caso de
Hamada et al.\cite{hamada}, el cual se ha adjudicado
el Gordon Bell prize~\footnote{\url{http://awards.acm.org/bell/}}
en variadas ocasiones, obteniendo en uno de sus últimos
trabajos obtener 190 TFlops utilizando algoritmos
TC en el problema de n-cuerpos.

\subsection{Multipole methods}
% n

El \emph{Fast Multipole Mehod} es uno de los métodos
más usados, por su poca complejidad y la alta precisión que provee,
siendo propuesto por Leslie Greengard en su tesis de doctorado~\cite{leslie}.

A grandes rasgos, consiste en un \emph{Treecode} que utilizar
dos representaciones del campo de potencial,
las cuales son, un campo lejano, que sería un \emph{multipole}
y expansiones locales.

Como éste método utiliza un cálculo rápido del campo de potencial,
es muy fácil poder realizar el procedimiento computacionalmente,
en comparación al cálculo de la fuerza, siendo esta última
un vector contrariamente al escalar que es el potencial $(\phi (x,y,z))$.
Aquí es necesario poder recordar una equivalencia de física
básica, donde la fuerza puede ser representada como el gradiente
negativo del potencial).

La estrategia del método es poder calcular una pequeña expresión
para el potencial, utilizando expansiones de multipolo, las
cuales se asemejan a una expansión de Taylor,
teniendo una alta precisión cuando se obtienes grandes
valores a partir de la expresión: $x^{2} + y^{2} + z^{2}$.

Como se comentaba anteriormente,
el presente método comparte la idea de árbol del método \emph{Treecode}
pero se diferencia en los siguientes aspectos:

\begin{itemize}
	\item El presente método calcula el potencial en todos los puntos,
		no la fuerza, como lo hace \emph{Treecode}.
	\item El presente método utiliza más información que la masa
		y el centro de las partículas en una región, ya que al ser
		una expansión con más precisión, crece su complejidad.
	\item La decisión de si una región (conjunto de cuerpos) se utiliza o no
		como un sólo cuerpo, depende sólo de la posición y tamaño
		de la región, no de la posición del centro de masa de la región.
\end{itemize}

El ser un método con más precisión, tiene su costo,
el cual fue demostrado por McMillan et al.\cite{mcmillan},
los cuales mencionan que no es un buen modelo para estudiar colisiones de sistemas
debido a que es más lento en comparación a otros métodos y
por que posee un tiempo constante al momento de evaluar la fuerza de un número
de partículas, con un orden O(1).

Sin embargo, el presente modelo es ampliamente utilizado para aplicaciones
en las cuales los cuerpos poseen los mismos o similares
pasos de tiempos, al momento de evaluar las ecuaciones de movimiento.

Si bien es cierto,
este algoritmo es famoso al proponer un orden usualmente de $O(n)$,
existe un trabajo de Srinivas Aluru~\cite{srinivas},
en la cual asegura que el algoritmo no posee dicho orden,
debido a que sólo en el mejor de los casos es obtenido,
dejando de lado el análisis del peor escenario posible para
el algoritmo. Al analizar cada paso del algoritmo,
desde la construcción del mismo árbol, en donde Greengard
establece algunas constantes en las cuales Srinivas
justifica que no son posible, se llega a la conclusión
de que el algoritmo a lo menos se comportará con un orden
de $O(N log^{2} N)$.
