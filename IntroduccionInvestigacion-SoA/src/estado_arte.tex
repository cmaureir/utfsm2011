% Criticar soluciones existentes, no agresivamente.
% 	Analisis -> Criterio (que usaremos para analizar nuestro propio trabajo)
% 	Definir criterios de evaluación.
% 	Explicar lo que implica y por que son importantes.

% Introduccion del estado del arte explicando enfoque

% formula de calculo de fuerza

El cálculo de la fuerza en su forma general está dado por,
$$f_{ij} =G \cdot \frac{m_i \cdot m_j}{||r_{ij}||^{2}} \cdot \frac{r_{ij}}{||r_ij||}$$,
donde las posiciones iniciales son $x_i$,
las velocidaddes son $v_i$,
teniendo a $i$, entre los valores, $1 \leq i \leq N$,
la masa de los cuerpos $i$ y $j$ determinada por $m_i$ y $m_j$,
siendo $r_{ij} = (x_j - x_i )$ vector de distancia entre los cuerpos $i$ y $j$
y finalmente $G$, constante gravitacional. ($6,67428 \times 10^{-11} m^{3} kg^{-1} s^{-2}$)

% Explicación de métodos para calculo fuerza

\subsection{Particle-Particle}

El presente enfoque es la forma más simple de poder abordar
la tarea del calculo de la fuerza que ejercen los cuerpos sobre otros.

A grandes rasgos, el método consiste en:
\begin{itemize}
	\item Acumular las fuerzas sobre una particula determinada (utilizando fórmula anteriormente señalada).
	\item Integrar las ecuaciones de movimiento.
	\item Actualizar el contador.
	\item Repetir el proceso.
\end{itemize}

Cabe señalar que el proceso en el cual se resuelven las ecuaciones,
consta de dos ecuaciones diferenciales de primer orden,
para poder calcular la aceleración y la velocidad,
y además se utiliza un método de integración, para obtener
las nuevas posiciones y velocidades de los cuerpos.

Como este panorama es el más sencillo,
podemos darnos cuenta de que el proceso es de orden $O(n^{2})$,
lo que para cualquier algoritmo, no es un escenario
deseable.

Al ser esta la solución incial, completamente teórica,
sin ningún tipo de mejora, cualquier texto que haga
referencia a métodos de integración será util,
pero también es conveniente recurrir a referencias
de simulación, como lo es el trabajo de Gould y Tabochnik~\red{methods}.

\subsection{Particle-Mesh}

Este método genera una malla en todo el espacio donde tenemos los cuerpos,
buscando así poder obtener la fuerza aproximandola en distintos
puntos de la malla.

Además, los operadores diferenciales del método pasado son reemplazados por
aproximaciones de diferencia finita, reduciendo el cálculo efectuado,
ya que la fuerza y los potenciales de las posiciones de los cuerpos
son obtenidos realizando una interpolación del arreglo de
los valores de la malla.

Otra característica importante de la malla, es que se define una especie
de densidad, las cuales son calculadas por las cargas que ejercen
cada cuerpo a un punto determinado de la malla.

A grandes rasgos, el método consiste en:
\begin{itemize}
	\item Asignar la carga a la malla, teniendo en cuenta la relación entre
		la masa del cuerpo y la densidad de la malla.
		\begin{itemize}
			\item La mejor forma será cuando variados cuerpos se encuentran
				cerca, para reducir las fluctuaciones de la fuerza.
		\end{itemize}
	\item Resolver la ecuación de potencia de campo sobre la malla,
		la cual puede ser resuelta utilizando la ecuación de Poisson: $\delta^{2} (\phi) = 4\cdot \pi \cdot G \cdot \rho$.
	\item Calculas la fuerza del campo a partir de los potenciales de la malla.
	\item Interpolar la fuerza sobre la malla para determinar las fuerzas
		sobre los cuerpos.
	\item Integrar las fuerzas para obtener las posiciones y velocidades
		de los cuerpos.
	\item Actualizar contador.
	\item Repetir el proceso.
\end{itemize}

Este método presenta un cálculo más rapido que el modelo \emph{Particle-Particle},
disminuyendo de la misma forma el orden del algoritmo a, $O(n + ng \log ng)$,
donde $ng$ es el número de vertices de la malla.

Un problema claro, es la complejidad que conlleva poder solucionar la ecuación de Poisson,
además de no ser un modelo recomendado para casos en que se quiera estudiar la colisión
entre cuerpos, ya que los trata casi como un sólo cuerpo, pues
aproxima cuerpos en los vértices de la malla.




%Edmund Bertschinger and James M. Gelb, "Cosmological N-Body Simulations," Computers in Physics, Mar/Apr 1991, pp 164-179.
%R.W. Hockney and J.W. Eastwood, Computer Simulation Using Particles, Institute of Physics Publishing, 1988
%A.L. Melott, Comment on "Nonlinear Gravitational Clustering in Cosmology", Physical Review Letters, 56, (1986), 1992.
%P.J.E. Peebles, A.L. Melott, M.R. Holmes, and L.R. Jiang, "A Model for the Formation of the Local Group," Ap J, 345, (1989) 108.

\subsection{Treecodes}

El presente modelo fue propuesto por Barnes \& Hut \red{CITE}
en el año \red{AÑO}, y hoy en día sigue siendo
uno de los modelos base más utilizado, pues a partir de este
mecanismo, muchos investigadores han realizado pequeñas modificaciones,
para poder obtener, ya sea un menor tiempo de ejecución o menor orden
del algoritmo.

El método se basa en la simple idea de que al momento de tener una cierta
cantidad de cuerpos bastante lejos, es posible aproximarlos todos
a un sólo cuerpo, determinando su posición a partir del centro de masa,
obteniendo un gran cuerpo, en vez de varios dispersos.

Además de ser un esquema inteligente para poder agrupar los cuerpos
lejanos, va dividiendo recursivamente todos los cuerpos,
en una estructura llamada \emph{quad-tree} en dos dimensiones,
y \emph{oct-tree} en tres dimensiones, los cuales analogamente
tienen una representación como un árbol, insertando la idea
de que la raíz del árbol representa el espacio completo,
y a medida que van agregandose hijos, estos representarán
divisiones dentro del espacio donde se encuentran.
Finalmente podremos decir que cada nodo externo,
representará un único cuerpo, mientra que los internos,
representarán el grupo de cuerpos aproximado.

El procedimiento para calcular la fuerza sobre un cuerpo consiste en:
\begin{itemize}
	\item Recorremos todos los nodos del árbol,
	\begin{itemize}
		\item Si es un nodo externo,
			utilizamos la misma idea del enfoque \emph{Particle-Particle},
		\item Si es un nodo interno,
		\begin{itemize}
			\item Si está lo suficientemente lejos, utilizamos la aproximación.
				Para determinar si el cuerpo está suficientemente lejos,
				se utiliza un cociente entre el ancho de la región donde está
				el nodo y la distancia del presente cuerpo al centro de masa.
				Dicho cociente se compara con un umbral que nos entregará la respuesta.
			\item Si no está lo suficientemente lejos, nos adentramos en los sub-árboles.
		\item Repetimos el proceso
		\end{itemize}
	\end{itemize}
\end{itemize}

El procedimiento para construir el árbol,
es decir, ingresar un nuevo cuerpo $b$ en el árbol
con raíz en un nodo $x$, consta en:
\begin{itemize}
	\item Si $x$ no tiene un cuerpo, $b$ se ingresa en dicha posición.
	\item Si $x$ es un nodo interno, actualizamos el centro de masa y la masa total.
		(Recursión, hasta que no sea un nodo interno).
	\item Si $x$ es un nodo externo, y contiene un nodo $c$,
		como habrían dos cuerpos en la misma región,
		se va subdividiendo recursivamente hasta que ambos cuerpos
		queden en regiones distintas.
\end{itemize}

El principal punto a favor,
es que las fuerzas sobre todos los cuerpos se obteienen con operaciones
con orden $O(n log n)$.

El problema es que se pierde precisión al aproximar los cuerpos
en uno sólo al estar lo suficientemente lejos.



%A wonderful explanation of the serial Barnes-Hut algorithm can be found on the Web at UC Berkeley as a course lecture. Go to: http://www.icsi.berkeley.edu/cs267/lecture26/lecture26.html. (For more information about this site, See the below Other N-Body WWW Sources.
%Another very nice explanation of Tree Codes can be found at: http://www.infomall.org/npac/pcw/node278.html from the book: Parallel Computing Works. This book describes a set of application and systems software research projects undertaken by the Caltech Concurrent Computation Program (CP) from 1983-1990.
%You can find several version of Barnes' Tree code at: ftp://hubble.ifa.hawaii.edu/pub/barnes/treecode/
%A.W. Appel, "An Efficient Program for Many-Body Simulations," Siam, J. Sci. Stat. Comput., 6 (1985), 85-103.
%J. Barnes and P. Hut, "A Hierarchical O(n log n) Force Calculation Algorithm," Nature, v. 324, December 1986.
%Joshua E. Barnes and Lars E. Hernquist, "Computer Models of Colliding Galaxies," Physics Today, March 1993, pp 54-61.
%J. Barnes and L. Hernquist, Annual Reviews of Astronomy and Astrophysics, 1992.
%L. Hernquist, ApJ Supplement, 64, 1987, 715.
%S. L. W. McMillan and S. J. Aarseth (1993). "An O(NlogN) Integration Scheme for Collisional Stellar Systems", Ap. J., Vol. 414, p. 200-212.
%Susanne Pfalzner and Paul Gibbon, Many-Body Tree Methods in Physics, 1996 176 pp. 49564-4 Hardback $57.95, Cambridge University Press. A clear description and introduction to tree-methods. I like it. You can get more information about the book from this Europe Web site and this U.S. site.
%Sellwood, J.A. 1989, MNRAS, 238, 115.
%J. Salmon and M. Warren, "Skeletons from the Treecode Closet", J. Comp. Phys. v 111, n 1, 1994; or http://www.ccsf.caltech.edu/~johns/ftp/nbody/skeletons.ps.Z
%For more Salmon bibliographic references related to Barnes-Hut, see: John Salmon's "Electronically available publications"
%The following report describes an implementation of Salmon's version of the Barnes-Hut algorithm to run on a the CM-5. The code is written so that it is relatively easy to port other tree algorithms (the Greengard/Rokhlin 3D adaptive FMM for example).
%P. Liu and S. Bhatt, Experiences with Parallel N-body simulations, to appear in the Proceedings of the 1994 ACM Symposium on Parallel Algorithms and Architectures.
%P. Liu, The parallel implementation of N-body algorithms, Ph.D. dissertation, Yale University, 1994.
%An N-body Tree method that I have not had time to look at is Andrey Kravtsov's "Adaptive Tree Refinement (ART)" (see publication list, which is a mesh based Eulerian code which adaptively refines meshes in the places where higher force resolution is desired. This method is used for large high-resolution cosmological simulation.



\subsection{Multipole methods}
% n

El \emph{Fast Multipole Mehod} es uno de los métodos
más usados, por su poca complejidad y la alta precisión que provee.

A grandes rasgos, consiste en un \emph{Treecode} que utilizar
dos representaciones del campo de potencial,
las cuales son, un campo lejano, que sería un \emph{multipole}
y expansiones locales.

Como éste método utiliza un cálculo rápido del campo de potencial,
es muy facil poder realizar el procedimiento computacionalmente,
en comparación al cálculo de la fuerza, siendo esta última
un vector contrariamente al escalar que es el potencial $(\phi (x,y,z))$.
Aquí es necesario poder recordar una equivalencia de física
básica, donde la fuerza puede ser representada como el gradiente
negativo del potencial).

La estrategia del método es poder calcular una pequeña expresión
para el potencial, utilizando expansiones de multipolo, las
cuales se asemejan a una expansión de Taylor,
teniendo una alta precisión cuando se obtienes grandes
valores a partir de la expresión: $x^{2} + y^{2} + z^{2}$.

Como se comentaba anteriormente,
el presente método comparte la idea de árbol del método \emph{Treecode}
pero se diferencia en los siguientes aspectos:

\begin{itemize}
	\item El presente método calcula el potencial en todos los puntos,
		no la fuerza, como lo hace \emph{Treecode}.
	\item El presente método utiliza más información que la masa
		y el centro de las particulas en una región, ya que al ser
		una expansión con más precisión, crece su complejidad.
	\item La desición de si una región (conjunto de cuerpos) se utiliza o no
		como un sólo cuerpo, depende sólo de la posición y tamaño
		de la región, no de la posición del centro de masa de la región.
\end{itemize}

El ser un método con más precisión, tiene su costo,
el cual fue demotrado por McMillan et al.\red{CITE}, % McMillan and S. J. Aarseth (1993)
los cuales mencionan que no es un buen modelo para estudiar colisiones de sistemas
debido a que es más lento en comparación a otros mpetodos y
por que posee un tiempo constante al momento de ebaluar la fuerza de un número
de particulas, con un orden O(1).

Sin embargo, el presente modelo es ampliamente utilizado para aplicaciones
en las cuales los cuerpos poseen los mismos o similares
pasos de tiempos, al momento de evaluar las ecuaciones de movimiento.

Si bien es cierto,
este algoritmo es famoso al proponer un orden usualmente de $O(n)$,
existe un trabajo de Srinivas Aluru \red{CITE},
en la cual asegura que el algoritmo no posee ese orden,
debido a que \red{RESPUESTA}.

% For a wonderful explanation (pictures etc.) see the UC Berkeley Web site: http://www.icsi.berkeley.edu/cs267/lecture27/lecture27.html. (Most of the information in this section came from there. See the below Other N-Body WWW Sources for more information about this site.)
% Srinivas Aluru, "Greengard's N-Body Algorithm Is Not Order N," SIAM Journal on Scientific Computing, May 1996, Volume 17, Number 3.
% William D. Elliott and John A. Board, Jr., "Fast Fourier Transform Accelerated Fast Multipole Algorithm," SIAM Journal on Scientific Computing, March 1996, Volume 17, Number 2.
% Leslie Greengard, "The Numerical Solution of the N-Body Problem, " Computers in Physics, Mar/Apr 1990, pp 142-152.
% This algorithm was first published in "A Fast Algorithm for Particle Simulations", L. Greengard and V. Rokhlin, J. Comp. Phys., v 73, 1987.
% Greengard's 1987 Yale dissertation "The Rapid Evaluation of Potential Fields in Particle Systems" won an ACM Distinguished Dissertation Award.
% S. L. W. McMillan and S. J. Aarseth (1993). "An O(NlogN) Integration Scheme for Collisional Stellar Systems", Astrophys. J., Vol. 414, p. 200-212.
% An adaptive FMM based upon Van Dommelen and Rundensteiner (J. Comp. Phys, v83, pp. 126-147 (1989)) has been implemented on a T8000-transputer system. It relies on sorting the vortices in the streamwise and transverse directions, and then distributing (approximately) equal numbers on each processor for even load balancing. This seems to work well, particularly when the problem granularity is coarse (ie: Large number of vortices, O(10^6), on relatively few processors, O(10). For more details, see the paper Clarke and Tutty (1994) below.
% Clarke, NR & Tutty, OR, "Construction and Validation of a Discrete Vortex Method for the Two-Dimensional Incompressible Navier-Stokes Equations," Computers Fluids, vol. 23, No. 6, pp 751-783 (1994).
% 
% A possible useful reference is G. Pringle's thesis: "Numerical Study of Three-Dimensional Flow using Fast Parallel Particle Algorithms" which describes a parallel version of the FMM in both 2 and 3 dimensions.
% You may find the following paper useful: "A comparison between Fast Multipole Algorithm and Tree-Code to evaluate gravitational forces in 3-D" by R. Capuzzo-Dolcetta and P. Miocchi, submitted to Journ. Comp. Phys. You can download it at: http://xxx.lanl.gov/abs/astro-ph/9703122.
% Here is more FMM related work: An Improved Fast Muiltipole Algorithm for Potential Fields, by Tomasz Hrycak and Vladimir Rokhlin.
% 
