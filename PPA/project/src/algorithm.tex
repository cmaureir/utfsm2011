\begin{frame}[fragile]
	\frametitle{Algorithm}
	\framesubtitle{Pseudocode}
	\begin{center}
		\begin{algorithmic}
			\STATE $updateAccelerations()$
			\FOR {$counter = 0 \to iterations$}
			    \STATE $updatePositions()$
			    \STATE $updateAccelerations()$
			    \STATE $updateVelocities()$
			\ENDFOR
		\end{algorithmic}
	\end{center}
\end{frame}

\begin{frame}[fragile]
    \frametitle{Algorithm}
    \framesubtitle{Profiling}
     \begin{itemize}
         \item Using \texttt{gprof}.
     \end{itemize}
     \begin{table}[h!t]
        \centering
        \scriptsize
        \begin{tabular}{|c|c|c|c|c|}
           \hline
           {\bf \% Time } & {\bf self seconds} & {\bf calls} & {\bf total ms/call} & {\bf name} \\\hline
             99.74   &  51.20   &  1002   &  51.10  & updateAccelerations() \\\hline
              0.08   &   0.04   &  1001   &   0.04  & updateVelocities()\\\hline
              0.02   &   0.01   &  1001   &   0.01  & updatePositions()\\\hline
              0.00   &   0.00   &     1   &   0.00  & checkAndInit(int, char const**)\\\hline
        \end{tabular}
        \caption{Serial implementation, using 1024 bodies}
     \end{table}
\end{frame}


\begin{frame}[fragile]
\frametitle{Algorithm}
\framesubtitle{OpenMP}
\begin{lstlisting}[style=C]
void updateAccelerations(){
    int i,j;
    for (i = 0; i < n; i++) {
        bodies[i].ax = bodies[i].ay = bodies[i].az = 0;
    }
    #pragma omp parallel for private(j)
    for (i = 0; i < n; i++) {
        for(j=0; j<n; j++){
            if(j != i)
                accelerationCalc(i,j);
        }
    }
}
\end{lstlisting}
\end{frame}

\begin{frame}[fragile]
\frametitle{Algorithm}
\framesubtitle{Pthreads}
\begin{lstlisting}[style=C]
void updateAccelerations(){
    int i;

    for (i = 0; i < n; i++) {
        bodies[i].ax = bodies[i].ay = bodies[i].az = 0;
    }
    int bound = n/cores;
    for (i = 0; i < cores; i++) {
        threads[i].ini = i * bound;
        threads[i].end = threads[i].ini + bound;
        pthread_create(&threads[i].thread, NULL , accelerationCalc, (void *)&threads[i]);
    }

    for(int j=0 ; j < cores ; ++j){
        pthread_join(threads[j].thread, NULL);
    }

}
\end{lstlisting}
\end{frame}

\begin{frame}[fragile]
\frametitle{Algorithm}
\framesubtitle{CUDA}
\begin{lstlisting}[style=C]
__device__ void update_accelerations(particle *d_bodies,int n){

    int i = threadIdx.x + blockDim.x * blockIdx.x;
    int j;

    if (i < n){
        for(j=0; j<n; j++){
            if(j != i)
                accelerationCalc(i,j);
        }
    }
}
\end{lstlisting}
\end{frame}


\begin{frame}[fragile]
\frametitle{Algorithm}
\framesubtitle{CUDA}

\begin{lstlisting}[style=C]
  nbody<<< blockNum, threadsPerBlock >>> (d_bodies,ite,dt,n);
  cudaThreadSynchronize();
\end{lstlisting}

\begin{lstlisting}[style=C]
__global__ void nbody(particle *d_bodies,float ite, float dt, int n){

    float t;
    reset_accelerations(d_bodies,n);
    __syncthreads();
    update_accelerations(d_bodies, n);
    __syncthreads();

    for (t = 0.0; t < ite; t += dt){
        update_positions(d_bodies,dt,n);
        __syncthreads();
        reset_accelerations(d_bodies,n);
        __syncthreads();
        update_accelerations(d_bodies,n);
        __syncthreads();
        update_velocities(d_bodies,dt,n);
        __syncthreads();
    }
}
\end{lstlisting}
\end{frame}
