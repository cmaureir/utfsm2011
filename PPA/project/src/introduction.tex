\frame
{
\frametitle{Trabajo}
\framesubtitle{Problema de n-cuerpos}
\begin{block}{Definición}
    Predecir el \underline{movimiento} de un grupo de \underline{objetos celestes},
    que van \underline{interactuando} unos con otros gravitacionalmente.
\end{block}
}

\frame
{
\frametitle{Trabajo}
\framesubtitle{Simulación de n-cuerpos}
\begin{itemize}
    \item Simulación de un sistema dinámico de partículas.
    \item Aplicaciones
    \begin{itemize}
        \item Formación de estructuras no lineales.
        \begin{itemize}
            \item Filamentos de galaxias.
            \item Halos de galaxias.
        \end{itemize}
        \item Dinámica de ``Star clusters''.
        \item Movimiento de planetas.
        \item etc.
    \end{itemize}
\end{itemize}
}


\frame
{
\frametitle{Trabajo}
\framesubtitle{Algoritmo}
Secciones críticas:
\begin{columns}
    \begin{column}{0.5\textwidth}
        \begin{itemize}
            \item Posiciones iniciales.
            \begin{itemize}
                \item \red{Random.}
                \item Plummer Model.
            \end{itemize}
            \item Método de integración.
            \begin{itemize}
                \item Euler.
                \item \red{Leapfrog.}
                \item Runge Kutta.
                \item Two step Adams-Bashworth.
                \item etc.
            \end{itemize}
        \end{itemize}
    \end{column}
    \begin{column}{0.5\textwidth}
        \begin{itemize}
            \item Cálculo de las fuerzas.
            \begin{itemize}
                \item \red{Particle-Particle.} \blue{$O(n^2)$}
                \item Particle-Mesh.  \blue{$O(n + ng\log ng)$}
                \item Treecode. \blue{$O(n\log n)$}
                \item Multipole. \blue{$O(n)$}
                \item etc.
            \end{itemize}
        \end{itemize}
    \end{column}
\end{columns}
}


\frame
{
\frametitle{Trabajo}
\framesubtitle{Algoritmo}

\begin{itemize}
    \item Cálculo de la fuerza.
    \begin{itemize}
        \item Posiciones iniciales $x_i$.
        \item Velocidades iniciales $v_i$.
        \item $1 \leq i \leq N$
    \end{itemize}
    $$f_{ij} =G \cdot \frac{m_i \cdot m_j}{||r_{ij}||^{2}} \cdot \frac{r_{ij}}{||r_ij||}$$
    \begin{itemize}
        \item $m_i$ y $m_j$ masas de los cuerpos $i$ y $j$.
        \item $r_{ij} = (x_j - x_i )$ vector de distancia entre los cuerpos $i$ y $j$.
        \item $G$, constante gravitacional. ($6,67428 \times 10^{-11} m^{3} kg^{-1} s^{-2}$)
    \end{itemize}
\end{itemize}
}



