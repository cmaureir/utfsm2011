\section{Conclusiones}

\textsc{Music Labs} dará oportunidades a muchas
bandas de la región, provocando un cambio radical
durante la etapa de maduración de las bandas emergentes.
La idea planteada está ligada a fomentar el desarrollo cultural
en la región, explotando un nicho que hasta el momento
sólo había sido atacado por sectores, ya que no existe
una oferta clara en este ambiente. Este desarrollo
también irá ligado directamente con nuevas
posibilidades de trabajo para músicos de la zona a través
de las distintas oportunidades que surgen a raíz
del trabajo que van realizando a medida que se
hacen más conocidos.


Hoy en día es más común ver nuevas bandas musicales gracias al mayor acceso a internet y a la disminución
de los precios de los instrumentos y equipos musicales.

No hay un registro histórico de las bandas que han ido surgiendo en nuestro país, por ende es muy difícil
recabar datos para hacer un análisis de demanda. Sin embargo se buscó por internet y se hizo una proyección
de demanda que arroja una tendencia  al crecimiento en la cantidad de bandas que se van formando año a año.

Según datos entregados por el club de música, nuestra universidad tiene 23 bandas, asumiendo que la cantidad de bandas es
proporcional a la cantidad de alumnos se estimó un total de 143 bandas en las universidades pertenecientes al consejo de rectores de
la región. Existen sólo 7 locales importantes que ofrecen servicios de sala de ensayo y/o estudio de grabación en la V región.

Se consultó a músicos de la zona y se concluyó que una banda en promedio trabaja 4 horas semanales en una sala de ensayo.

 
La V región tiene características deseables para llevar a cabo el proyecto, como por ejemplo tener una
gran cantidad de estudiantes y de personas pertenecientes al nivel socioeconómico C2 y C3. 



Se ha realizado un primer análisis con respecto a varios
aspectos del proyecto, que lo deja mejor
posicionado para enfrentar los retos que irán apareciendo en
el camino.

Debido al claro crecimiento de las bandas emergentes
tanto en la región como el país, es evidente
el nivel de importancia que tiene la evolución musical que se viene 
gestando desde algunos años en Chile.

A través de los servicios que \textsc{Music Labs} ofrece,
se está haciendo un llamado de atención
a las diferentes fuentes de apoyo a la música nacional
para dar a conocer el potencial artístico que tiene la zona.

Se ha generado además un estudio inicial con respecto
a las situaciones relacionadas con el proyecto,
sin dejar de lado un análisis fundamental que era ver
la separabilidad del proyecto, lo que impulsó
la idea de tener un servicio principal
compuesto de otros servicios más particulares.

Por otro lado fue importante listar y describir
los principales elementos para dimensionar los costos
y beneficios que poseía el proyecto. 

El hecho de realizar un análisis de la demanda y oferta, ilustra
el comportamiento de las bandas en constante crecimiento, por lo que
satisfacer las necesidades de manera íntegra es una oportunidad valiosa.
