\section{Introducción}
La industria de la música constituye hoy en día uno de los sectores de mayor
crecimiento de la economía mundial. Las nuevas tecnologías han acercado la música
al consumidor a tal punto de que sea posible comprar un disco sin la necesidad
de moverse del hogar, a través de los servicios disponibles en Internet.

En Chile existe una gran cantidad de bandas musicales emergentes que si bien existen, no han podido dar a conocer su trabajo a nivel nacional o mundial ya que
tienen la  necesidad de contar con un servicio que brinde un soporte completo para hallar el camino al éxito. 
Entre estos servicios se encuentran:
Sala de ensayo, estudio de grabación, gestión de eventos, mánager y
posicionamiento en las principales redes sociales del mundo.

Muchas bandas no logran el ansiado éxito debido a que la mayoría de las 
salas de ensayo y estudios de grabación se encuentran en Santiago y generalmente
se ofrecen ambos servicios por separado, siendo la gestión en esta primera etapa, se presenta el diagnóstico de la idea y el estudio de mercado estratégica de la banda 
responsabilidad de ella misma. 

Debido a esto se propone una solución que además
de entregar el espacio físico donde una banda se pueda desarrollar, entrega el 
apoyo en términos de gestión de los distintos procesos por la cual debe pasar, 
mediante la incorporación de equipos de trabajo multidisciplinarios que darán soporte
desde la grabación del disco hasta el posicionamiento en los principales 
medios de comunicación en el país y el mundo.
