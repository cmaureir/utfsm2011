\section{Diagnóstico}

\subsection{Definición de la Idea del Proyecto}
La idea general que desarrolla este proyecto está centrada en la necesidad 
actual de la gran mayoría de las bandas musicales emergentes. Una agrupación 
musical requiere un lugar que le brinde el soporte técnico necesario 
para componer y ensayar su música. En una etapa previa la banda 
requiere grabar sus composiciones para dejar registro y poder hacer 
conocido públicamente su trabajo. \\

Lamentablemente, muchas bandas se quedan estancadas en esta etapa, ya 
que no tienen posibilidades concretas de mostrar su trabajo al 
público. Un público que si bien existe, no tiene la oportunidad de 
conocer nuevas bandas por las pocas opciones de difusión con que 
cuentan las bandas emergentes.\\

Este proyecto está enfocado en crear una instancia que permita a las 
bandas nuevas, componer, grabar y difundir su trabajo musical. 

\subsection{Objetivos del proyecto}
	%\subsubsection{Objetivo General}
	Entregar un servicio diferenciado y multidisciplinario 
	que integre los distintos procesos por los cuales 
	debe pasar una banda musical emergente al momento 
	de lanzar su carrera. 
	
	Posicionar a \emph{Music Labs} como el mejor
	proveedor de este servicio en la región, siendo una incubadora
	de bandas emergentes que demuestre
	entregar además de los espacios físicos,
	la asesoría necesaria para dar soporte a las distintas demandas de las 
	bandas musicales actuales.

	Utilizar los medios electrónicos actuales
	para dar a conocer a las distintas bandas
	emergentes que utilicen el servicio,
	a nivel regional, nacional y mundial

	
	
	%Además, entregar asesorías en cada uno de estos 
	%procesos a través de profesionales capaces de 
	%guiar y evaluar el trabajo realizado por las 
	%distintas bandas cubriendo las distintas necesidades 
	%de ellas.

	%\subsubsection{Objetivos Específicos}
	%Crear un completo servicio de apoyo al lanzamiento de una 
	%banda musical que cuente con lo siguiente:
	%	\begin{itemize}
	%		\item Sala de ensayo con los implementos 
	%			necesarios que permita a una banda musical 
	%			desarrollarse profesionalmente.
	%		\item Estudio de grabación que cuente con la presencia 
	%			de profesionales expertos en grabación y producción 
	%		\item Salón de eventos, para poder crear una primera instancia
	%			de un espectáculo abierto al público de las nuevas bandas.
	%		\item Productora de eventos, cuya misión será la organización 
	%			y difusión de encuentros entre las distintas bandas.
	%		\item Managers a la disposición de las bandas que lo requieran.
	%		\item Difusión de las bandas a través de Internet lo cual 
	%			contempla la creación de una página Web por cada banda 
	%			y el posicionamiento de esta en las redes sociales 
	%			más populares como Facebook y Twitter.
	%	\end{itemize}

\subsection{Antecedentes generales del proyecto}

El mayor problema que enfrenta una agrupación o banda musical al momento
de comenzar una carrera es encontrar la asesoría adecuada para iniciar
una carrera rentable en este mercado competitivo. La primera dificultad
es encontrar una adecuada sala de ensayo que cuente con todos los 
implementos necesarios para que las bandas mejoren su calidad musical. 
Una vez encontrada una buena sala de ensayo, algunas bandas deciden grabar
su primer disco y entrar en el mercado. Las bandas de regiones se encuentran
con el hecho de que los mejores estudios de grabación se encuentran en Santiago.

Por otra parte, la difusión de la banda en muchos casos es responsabilidad
de los mismos integrantes. Entre estas tareas se encuentra buscar y utilizar
los medios para poder difundir su música.

Concretamente, existen pocos eventos donde se dé el espacio a una banda 
emergente de presentar su música para ser conocidos a nivel nacional
o internacional. Este punto es un riesgo que asume la banda al lanzar
su carrera.

Es por esto que actualmente en Chile existe una necesidad para las bandas 
emergentes de contar con un servicio integral que cuente con los implementos
necesarios que permitan que las bandas ensayen sus canciones, graben sus discos
y finalmente difundan su trabajo para entrar en el competitivo mercado
de la música. La distribución de estos recursos es totalmente centralizada. En 
Chile existe un total de 175 salas de ensayo, de las cuales un 92\% de ellas 
se encuentran en la ciudad de Santiago, mientras que el porcentaje
restante se distribuye a lo largo del resto del país.

Con respecto a los estudios de grabación, la distribución es similar, con un
84,6\% de ellos ubicados en Santiago y el resto en regiones. 

En cuanto al apoyo en difusión y eventos, ninguno de estos servicios antes 
mencionados cuenta con un apoyo integral a las bandas.

\subsection{Alcance del proyecto}

El proyecto se realizará en primera instancia 
en la V región y estará enfocado a bandas emergentes 
que necesiten soporte musical para ensayar y 
componer su trabajo. Además obtendrán apoyo 
en difusión mediante eventos organizados 
especialmente para apoyar el inicio de todas 
las bandas que utilicen el servicio.\\

Con respecto a los eventos, estos se realizarán en 
un comienzo en la región mediante convenios 
con locales asociados al proyecto y en un centro de eventos
propio  del servicio ubicado en la V Región. La frecuencia 
de estos están sujetos a la cantidad de locales 
asociados y su disponibilidad.\\

La difusión se realizará en 
las distintas redes sociales más populares del 
momento y su éxito será medido de acuerdo a 
la cantidad de personas que visiten la información 
de la banda.\\

Con respecto al servicio, la banda musical 
que acceda al servicio, contará con los implementos 
necesarios para realizar sus ensayos, grabar sus 
discos, actuar en eventos regionales o nacionales 
y el respaldo tecnológico para su difusión en 
el mundo a través de Internet, para así alcanzar 
el éxito.\\

La cantidad de Managers disponibles dependerá de los
contactos que maneje el proyecto y las bandas podrán 
solicitar la información de contacto de estos.\\

Cabe destacar que, si bien, algunos estudios 
de grabación incluyen sala de ensayo, ninguno 
de ellos ofrece un servicio completo de 
difusión y asesoría para las bandas musicales 
lo cual hace que el proyecto planteado tenga 
una ventaja competitiva, ya que está 
orientado al cliente en todo aspecto.

\subsection{Justificación del proyecto}

Del 100\% de salas de ensayo que se encuentran en el país (172), el 92\% 
se concentra en la región metropolitana. De las restantes, sólo 7 salas se 
encuentran en la V región (correspondiente al 4\%). Si se utilizan los
datos demográficos obtenidos desde los resultados del censo realizado 
en el año 2002 en Chile, se puede observar que, si bien el 40\% de la 
población total del país se concentra en la región metropolitana, más
del 20\% se encuentra repartido entre la V y la VIII región. Esto indica que, 
de existir una relación equitativa entre las salas de ensayo y estudios de
grabación, y la población total, al menos el 20\% del total de salas de 
éstas deberían existir entre estas regiones. Pero, ¿Por qué realizar 
esta comparación ? La razón es muy simple. Las personas que tienden a 
formar grupos de música, realizar tocatas, etc, son el público adolescente
 - adulto joven. Esto incluye mayormente a los estudiantes de educación media
y universitarios (personas entre 15 y 30 años mayormente). La zona de la V 
región es conocida por la cantidad de estudiantes que posee. De hecho, 
se puede encontrar 4 universidades pertenecientes al consejo de rectores, de
gran prestigio (lo que implica que gran cantidad de estudiantes se acercará
a estas instituciones), además de una serie de universidades privadas 
líderes en cada uno de sus ámbitos, y una serie de Institutos de Educación
superior y Centros de Formación Técnica. Por lo tanto, existe una alta
presencia de personas a las cuales ofrecer el nuevo servicio que se propone.

\subsection{Impacto del proyecto}

Chile posee una gran diversidad musical y 
cultural que perdura en el tiempo mediante la 
composición nueva música y la reversión de 
temas que ya son parte del mundo cultural 
de Chile. Este proyecto permitirá aumentar 
el abanico de bandas y solistas que trabajan 
en el mundo de la música y que no tienen 
las mismas oportunidades de difusión que 
las grandes bandas. Esto, a través de un 
servicio de calidad y costo apropiado según 
la realidad de las nuevas bandas. Además, al
contar con mayor diversidad de bandas es posible
aportar a la difusión cultural de la música
del país, ayudando a que la generación actual
y futura conozca la música que se hizo y que
se está haciendo a nivel local y nacional.\\ %(IMPACTO CULTURAL)

En la actualidad, el precio de los instrumentos 
musicales ha disminuido considerablemente respecto 
a décadas anteriores. Esto ha permitido que muchas 
personas tengan la oportunidad de adquirir instrumentos 
y tener como hobbie la ejecución musical. Es por esto 
que la cantidad de bandas emergentes también aumentó 
con el paso del tiempo. Este proyecto permite a las 
nuevas bandas ir un paso más adelante, dejando de 
tener una banda como pasatiempo y pasar a ser 
una banda profesional. Así. la idea acá presentada 
tendrá gran impacto social, al ser una nueva 
fuente de trabajo y desarrollo social y cultural 
para muchas personas.\\ %(IMPACTO SOCIAL)

Es posible visualizar un impacto negativo %IMPACTO ACÚSTICO A LOS VECINOS.
relacionado con la contaminación acústica
generada por alto nivel de sonido que emanará
del salón de eventos. El salón, al estar
ubicado en un sector céntrico de la ciudad
es posible que cause molestia a las personas
que vivan en los alrededores, no solo debido
al sonido sino también a los efectos producidos
por la gente que asiste a los eventos.\\

A nivel local, el proyecto tendrá cierto impacto
en el nivel de empleabilidad de la región, ya que
para la implementación será necesario contar con
distintos profesionales encargados de las distintas
áreas abarcadas por \emph{Music Labs}. Hay que destacar
que no será un impacto mayor por no ser necesario una
alta cantidad de trabajadores, pero si se convierte
en una fuente de trabajo para la región.
