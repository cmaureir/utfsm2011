\section{Resumen Ejecutivo}
Mediante el presente proyecto presentado por \textsc{Music Labs}
se evaluará la creación de un centro multidisciplinario que integre los
distintos procesos por los cuales debe pasar una banda emergente alcanzar el éxito.

% En primera instancia el servicio será ofrecido en la V región, para luego evaluar
%la posibilidad de expansión en la distintas regiones de Chile.

\textsc{Music Labs} será un servicio que reunirá todos los elementos para lanzar
una banda musical emergente a la fama. Entre estos elementos se encuentran:
estudio de grabación, sala de ensayo, salón de eventos, asesoría personal 
y posicionamiento en la Web 2.0. Esto se justifica en el hecho que actualmente
en Chile no existe un servicio que integre todos estos procesos en un solo servicio.

El estudio de mercado muestra que existe un aumento en la cantidad de 
bandas emergentes en la V región, esto motivado principalmente por dos factores a destacar:
La disminución de los valores de instrumentos y la masificación de los medios de 
comunicación como Internet. Además el estudio muestra que el segmento
objetivo se compone de la población adulto-joven, específicamente personas mayores 
de 18 años pertenecientes a los grupos socioeconómicos C2 y C3 los cuales representan 
el $55\%$ del mercado de la V región.

El análisis de la oferta indica que los principales competidores son salas de 
ensayo tradicionales y estudios de grabación. Generalmente estos servicios se 
ofrecen por separado. El $92\%$ de las salas de ensayo se concentra en la región
Metropolitana, de las restantes sólo el $4\%$ se encuentra en la V región.
Con respecto a los estudios de grabación el $85.2\%$ se encuentra en la R.M, mientras
que sólo un $5.8\%$ se encuentra en la V región.

 Por lo tanto, \textsc{Music Labs} se presenta como un proyecto innovador 
que posee una gran ventaja competitiva por ser un servicio
orientado al cliente apoyado por un equipo de profesionales altamente capacitados.

