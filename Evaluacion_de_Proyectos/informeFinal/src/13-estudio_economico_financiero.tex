\section{Estudio Económico Financiero}
    \subsection{Definición de Parámetros}
    	\subsubsection{Determinación de la tasa de descuento}
            El objectivo de \emph{Music Labs} es representar el retorno mínimo exigido por
            el inversionista para acceder a invertir en un proyecto.
                    
            La determinación de este parámetro se basa en el riesgo estimado inherente
            a la realización del proyecto y en la tasa anual de crédito de largo plazo
            solicitado para el financiamiento del mismo.
           
            El riesgo del proyecto determinado al ingresar con un nuevo proyecto
            al mercado, siendo un servicio innovador en la V región, se estima en un
            \red{X\%}.

            Además, para obtener los fondos iniciales para crear nuestro proyecto,
            será necesario solicitar un crédito a largo plazo con una tasa mensual de \red{Y\%}
            por lo que anualmente se estima una tasa de \red{Y*12\%}.

            Finalmente, la tasa de descuento es igual a un \red{X+Y*12\%}.
            %Como es un proyecto que quizás no existe se deben buscar proyectos "similares"
            %y ver su tasa de riesgo, y de acuerdo a la tasa anual del crédito, se suman
            %y da la tasa de descuento.

	    \subsubsection{Criterio para determinar horizonte de evaluación}
            El horizonte de evaluación hace referencia al lapso de tiempo
            que se quiere evaluar cómo será el funcionamiento de la empresa,
            en este caso  será un horizonte de \red{X años}, debido 
            a la permanencia en el mercado que se ha proyectado calcular
            la rentabilidad luego de una cantidad de años significativa.

            Este criterio puede ir ligado al calendario de reinversión
            ya que si el proyecto es rentable se debería reinvertir en los
            elementos fundamentales para su financiamiento.
        
    \subsection{Calendario de Montos de Inversiones y reinversiones}
    \subsection{Calendario de Montos de Ingresos}
    \subsection{Calendario de Montos de Egresos}
    \subsection{Depreciaciones}
    \subsection{Análisis de Proyecto Puro}
    	\subsubsection{Escudo Fiscal}
    	\subsubsection{Ventas de activos}
    	\subsubsection{Balance de IVA}
    	\subsubsection{Flujo de Caja Proyecto Puro}
    	\subsubsection{Obtención de Indicadores Proyecto Puro (VAN, TIR, PAYBACK)}
    \subsection{Análisis de Proyecto Financiado}
    	\subsubsection{Análisis de alternativa de financiamiento}
    	\subsubsection{Determinación de cuadro de amortización e intereses}
    	\subsubsection{Escudo Fiscal}
    	\subsubsection{Ventas de activos}
    	\subsubsection{Balance de IVA}
    	\subsubsection{Flujo de Caja Proyecto Financiado}
    	\subsubsection{Obtención de Indicadores Proyecto Financiado (VAN, TIR, PAYBACK)}
    \subsection{Análisis de Sensibilidad}
    	\subsubsection{Unidimensional}
    	\subsubsection{Análisis de los resultados y conclusiones}
    
