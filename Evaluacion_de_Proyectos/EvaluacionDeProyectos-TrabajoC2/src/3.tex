\section{Tercera pregunta}

Del estudio financiero averigüe los siguientes aspectos (15p)

\begin{enumerate}[(a)]
    \item {\bf Averigüe las fuentes de financiamiento a través de intermediarios
          (bancos, CORFO, sercotec, etc) Explique en qué consisten y sus características.}\\

          \red{Respuesta:}

        \begin{itemize}
               \item Subvención a la Asistencia Crediticia (SUAF-CORFO).\\

                    Consiste en la contratación de un consultor quien recopila y estandariza
                    la información financiera para ser presentada a la banca o compañías de leasing.

                \item Cupones de bonificación de primas de seguros de créditos (CUBOS).\\

                    Es una cobertura parcial de seguro por no pago del crédito, mediante el cual la Corporación
                    de Fomento (CORFO) subsidia el valor de las primas de seguro de crédito de operaciones de
                    pequeñas empresas y, es por tanto, una forma de resolver el problema de falta de garantías
                    reales por parte de la PYME.

                \item Líneas de financiamiento FONTEC.\\

                    El Fondo de Desarrollo Tecnológico y Productivo (FONTEC) es un fondo concursable al
                    que pueden postular proyectos de investigación y desarrollo en el ámbito tecnológico y
                    productivo.

                \item SERCOTEC~\footnote{\url{http://www.sercotec.cl/}}

                    El Servicio de Cooperación Técnica o más conocido como SERCOTEC es una institución chilena que apoya
                    iniciativas de mejoramiento de la competitividad de los micro y pequeñas empresas para fortalecer
                    la capacidad de gestión de las empresas y de los empresarios de estas. SERCOTEC es una filial de la Corporación de Fomento de la Producción (CORFO)
                   que entrega servicios de asesoría, capacitación y fondos (capital semilla). 


                \item CORFO~\footnote{\url{http://www.corfo.cl/}}

                    La Corporación de Fomento de la Producción (CORFO) es un ente estatal chileno el cual se encarga en la
                    actualidad de fomentar e innovar en el país mediante programas de asociatividad entre empresas,
                    el mejoramiento de la calidad y la productividad, promover la innovación tecnológica  y acceso
                    al financiamiento de empresas privadas en los sectores de mayor potencial económico en el país,
                    tales como el turismo de intereses especiales, la agroindustria,  minería y acuicultura entre otros.
                    Por otro lado, se encarga además de atraer selectivamente empresas e innovaciones extranjeras para que estas se instalen en el país a través de planes de incentivo.

                \item INDAP~\footnote{\url{http://www.indap.gob.cl/}}

                    El Instituto de Desarrollo Agropecuario (INDAP) es un servicio público chileno, que depende del Ministerio de Agricultura.
                    El cual tiene como objetivo principal  fomentar y apoyar la agricultura familiar campesina y de pequeños productores,
                    promover el desarrollo tecnológico del sector para mejorar su capacidad empresarial, comercial y organizacional.
                    Con lo anterior mencionado INDAP pretende erradicar la pobreza rural, en donde generaran políticas de desarrollo
                    sustentable y así integrar al campesinado en el crecimiento económico, modernizando sus prácticas y tecnología.

                \item FOSIS~\footnote{\url{http://www.fosis.cl}}

                    El Fondo de Solidaridad e Inversión Social (Fosis) es aquel que permite a los interesados en postular a proyectos convocados
                    por alguno de los programas de la institución en cuestión, la posibilidad de retirar o adquirir las bases de licitación para
                    postular al proyecto. Uno de sus principales objetivos es contribuir al país ayudando a la superación de la pobreza a través
                    del esfuerzo y aportando enfoques de trabajo complementarios a los que abordan otros servicios del Estado.

                    Además el FOSIS tiene los siguientes objetivos secundarios.

                    \begin{itemize}
                        \item La planificación de la asignación de recursos de inversión a través de la intervención territorial y gestión institucional.
                        \item La gestión de la Inversión de los diversos grupos beneficiarios.
                        \item La evaluación de la planificación y gestión de la inversión FOSIS, midiendo los resultados.
                        \item La innovación, buscando estrategias que eficientemente ayuden a realizar proyectos y resolver problemas.
                        \item etc.
                    \end{itemize}
        \end{itemize}


    \newpage 
    \item {\bf Averigüe las fuentes de financiamiento a través de inversionistas (accionistas).
          Explique en qué consisten y sus características.}\\

          \red{Respuesta:}

            En Chile es posible obtener financiamiento de inversionistas chilenos por préstamos comerciales, capital de riesgo y 
            financiamiento de proveedores. Sin embargo, es prudente recalcar que las empresas latinas tienen acceso a las fuentes 
            internacionales. Cuando se va a buscar financiamiento hay que seleccionar con mucho más cuidado cuál es el inversionista 
            que mayor afinidad tiene con la empresa y como va a ser la estructura de negociación.
           
            \begin{itemize}
                \item Inversionistas Angeles (business angels)
            
                    Invierte en las personas, a un riesgo muy alto, por lo que se convierte en un apoyo administrativo y gerencial difícil de reemplazar.

                    El inversionista ángel sólo invierte en empresas donde siente que puede aportar un elemento de dirección importante pues ya 
                    conoce la industria. Para una empresa en formación, el inversionista ángel constituye un valuarte insustituible pues brinda su 
                    experiencia y sus contactos además del capital. El llamado ``socio capitalista'' que no se involucra en las decisiones de la empresa 
                    que financia, no es un inversionista ángel.
            
                \item  Inversionistas institucionales
            
                    Existen numerosas modalidades, desde la liberación del pago de impuesto o la protección a ciertas industrias hasta el otorgamiento 
                    de dinero directo. Existe en Chile la modalidad de los Profo, o programas de asociatividad, en los cuales varias empresas similares 
                    se unen para afrontar la competencia juntas, desarrollar nuevos mercados o darse apoyo comercial de otra índole. En este caso, 
                    CORFO aporta la mitad del capital requerido para los proyectos. El trámite es relativamente sencillo. CORFO también brinda apoyo a través 
                    de sus líneas de financiamiento de innovación tecnológica: FONTEC.
            \end{itemize}


    \newpage
    \item {\bf ¿Qué factores se deben considerar al momento de evaluar la elección de una
          fuente de financiamiento a través de intermediarios o una fuente de financiamiento
          a través de inversionistas? Compare además las ventajas y desventajas entre ellas}\\

          \red{Respuesta:}

            Es difícil decidir por uno u otro tipo de financiamiento debido a que se debe tener claro lo siguiente:
           
            \begin{enumerate} 
                \item Cómo ha determinado la cantidad
                \item El propósito o uso que le dará
                \item La duración o método de pago
                \item La contribución económica que hará
                \item La viabilidad
                \item Qué activos tiene y su liquidez
                \item Cómo se administrará
            \end{enumerate}
            
            Una vez claro los puntos anteriores se puede acortar la lista para elegir a los financiadores posibles. 

            Si se tiene especificado el tipo de negocio y se ha hecho un estudio de mercado que indique que es viable a largo plazo,
            CORFO es un excelente financiador ya que aportara un porcentaje del capital requerido sin necesidad de endeudamiento, otra forma de
            financiar su negocio sin endeudamiento sería a través de los business angels, pero debe tener en cuenta que ellos buscaran si bien
            un alto riesgo, también alto retorno, por lo tanto se debe considerar que este inversionista además de aportar capital estará presente
            en las decisiones que involucren al negocio y/o empresa.

            Ahora bien si se está dispuesto a endeudamiento una opción son los intermediarios bancarios el problema es que las tasas de intereses 
            son altas y se debe cotizar todas las opciones disponibles antes de tomar la decisión.

            

\end{enumerate}
