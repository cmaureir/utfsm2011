\section{Resumen Ejecutivo}

	Las bandas emergentes deben sortear muchos obstáculos durante su desarrollo, entre los cuales se tienen el alto costo que les significa recurrir a todos 
	los servicios que necesitan, el depender de diversas empresas para la satisfacción de sus necesidades y la falta de orientación y contactos.
	
	En las siguientes secciones se analiza un proyecto que busca entregar una solución concreta y diferenciada de lo que existe en el mercado, 
	para satisfacer las necesidades anteriormente mencionadas.

	En la sección estudio de mercado se hace un análisis de las Cinco Fuerzas de Porter, en donde se determina que las principales amenazas para 
	el proyecto son los productos sustitutos y la principal barrera de entrada
es la necesidad de tener mucha experiencia y contactos en el mundo de la música
para ser capaz de resolver los problemas anteriormente planteados. También se
hace un análisis FODA al proyecto, encontrando que las principales debilidades
son la inversión requerida para su ejecución y el bajo nivel de dependencia de
los clientes con el servicio, debido a la posibilidad de suplirlo mediante
otros medios. La principal fortaleza del proyecto es que se ofrece a los
clientes un servicio completo, que actualmente sólo se puede conseguir
recurriendo a múltiples instancias. En esta sección también es posible
encontrar el segmento objetivo en donde se establece que se apuntará a personas
mayores de 18 años que se dediquen a la música parcial o totalmente y que
pertenezcan al nivel económico C2. Además se detalla una descripción acabada
del producto, marketing estratégico y marketing operativo.
% TO DO
%\red{ Falta incluir información de esos puntos (marketing estratéfico y marketing
% operativo}


	En la sección estudio técnico se detallan las decisiones y recursos necesarios para llevar a cabo el proyecto.
	 Se define el tamaño del proyecto, estableciéndose un horario de atención
entre las 09:00 y 22:00 horas con flujo máximo de 26 bandas diarias que hagan
	 uso de la sala de grabación y sala de ensayo.
	 La macrolocalización del proyecto fue determinada mediante cinco factores, 
	 entre los que se encuentran por ejemplo la cantidad de oferta por región y la fracción del segmento objetivo, 
	concluyendo que la V región es la más apta para un proyecto de este tipo debido a tener una gran cantidad de universidades, vida nocturna y eventos, 
	entre otros factores deseables. La microlocalización del proyecto incluye el análisis de diversos locales para la instalación y funcionamiento de Music Labs,
	 se optó por una casona ubicada en calle Alvarez, en sector céntrico de Viña del Mar que tiene un valor de arriendo de 1 millón de pesos mensuales. 
	En ingeniería del proyecto se encuentra una descripción detallada del proceso productivo, de los costos y equipos necesarios sumando una inversión de aproximadamente 16 millones de pesos en estos últimos. 

% TO DO 	
% \red{Faltan más datos cuantitativos}
