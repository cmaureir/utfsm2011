\section{Estudio de Mercado}

\subsection{Análisis de Sistema de Comercialización}

%blah blah
En la actualidad, las bandas emergentes deben recurrir a múltiples
instancias para satisfacer sus necesidades de ensayo, grabación y asesoría,
esto implica un esfuerzo mayor por parte de la banda en organizar sus diversas actividades.

%dice que el sistema es un desafío
Por el mismo motivo, un servicio que integre salas de ensayo, estudios
 de grabación. asesoría profesional, etc, se diferencia de los servicios ya existentes.

%.... repite lo mismo del párrafo anterior y agrega que debe tener publicidad
%clara
Dado que es un servicio nuevo, que integra características ya conocidas, 
la publicidad debe ser bastante clara en destacar dicha integración, de 
manera de dar a conocer los beneficios que esto trae, tanto para los 
consumidores menos conocedores del tema, como los expertos.
	
\subsubsection{Marketing Estratégico}

%blah blah =)
El proyecto, como se definió anteriormente, tiene como objetivo la 
integración de distintos servicios para bandas emergentes, por lo que 
un buen análisis y planteamiento de cómo se llevará a cabo, implica 
análisis tanto internos como externos.

\begin{itemize}

\item{Análisis Externo}


 Una de las herramientas más poderosas para un análisis externo es el conjunto de las cinco fuerzas de Porter:

\begin{itemize}
	\item Amenaza de nuevos participantes
	\item Amenaza de los productos sustitutos
	\item Poder negociador de los clientes
	\item Poder negociador de los proveedores
	\item Intensidad y rivalidad entre empresas existentes
\end{itemize}

\item Amenaza de nuevos participantes

    Es posible que entren nuevos actores al mercado que se quiere explotar, por ejemplo una sala de ensayo podría comenzar a ofrecer servicios similares, ya que la barrera de entrada en este caso no sería muy alta, por contar con la infraestructura y mayor conocimiento del negocio.

\vspace{0.5cm}
\begin{center}
\begin{tabular}{|c|c|c|c|c|c|}
\hline
Barrera de entrada          & Muy baja & baja & intermedia & alta & muy alta \\
\hline
Requerimiento de capital    &          &      & x          &      & \\
\hline
Diferenciación del producto &          & x    &            &      & \\
\hline
Experiencia                 &          &      &            & x    & \\
\hline
Legal                       &          & x    &            &      & \\
\hline
\end{tabular}
\end{center}
\vspace{0.5cm}
% TO DO
%\red{si se va incluir esta tabla, debe ser para todos los puntos.}

% TO DO
%\red{
%    - cambiar servicio que se quiere ofrecer por industria (y en otras partes
%      que hacen referencia a lo mismo).
%    - Resumen de las 5 fuerzas de potter: están explicadas de forma muy
%      general, falta conclusión de la industria y el análisis se realiza del
%punto de vista de la industria.
%        -  ¿Logística de salida?
%        - ¿Servicio post-venta?
%        - en oportunidades:
%            - las dos últimas son fortalezas.
%}
 Una de las principales barreras de entrada para un nuevo competidor es la experiencia y los contactos que este tenga con bandas importantes de la región, esto debido a que este tipo de servicio requiere de mucha cercanía con los clientes.

 
\item Amenaza de productos sustitutos

	El servicio que se quiere ofrecer tiene sustitutos, pues una banda puede satisfacer las mismas necesidades recurriendo a diversas empresas, por lo tanto esta es una amenaza importante para el proyecto. 

	Entre los diversos servicios que entran en la categoría de sustitutos para este proyecto están las salas de ensayo, estudios de grabación, managers de bandas y agencias de publicidad.

\item Poder de negociación de los clientes

	El mercado esta lleno de alternativas, por ende una banda puede desligarse fácilmente del servicio ofrecido. Además las bandas pasan por diversas crisis durante su desarrollo, factor que sin duda puede producir un efecto negativo.

\item Poder de negociación de los proveedores

	Los insumos necesarios para el funcionamiento del proyecto están en un mercado muy competitivo, por ende no hay una dependencia fuerte hacia los proveedores.

\item Intensidad y rivalidad entre empresas existentes

	Las empresas proveedoras de los distintos servicios ofrecidos tienen distintos niveles y ámbitos de rivalidad, por ejemplo las salas de ensayo no compiten con los estudios de grabación, ni los managers. 

	Music Labs quiere entrar al mercado ofreciendo un producto diferenciado, disminuyendo así la posibilidad de que lleguen competidores directos, pues los packs de servicios que se quieren ofrecer no existen en el mercado.
		
\item Resumen 5 Fuerzas de Porter

\begin{table}[h]
\centering
	\begin{tabular}{|c|c|c|c|c|c|}
		\hline
		                                                 & MPA & PA & N & A & MA \\
		\hline
		Amenaza de nuevos participantes                  &     &    & x &   & \\
		\hline
		Amenaza de los productos sustitutos              &     &    &   & x & \\
		\hline
		Poder negociador de los clientes                 &     &    & x &   & \\
		\hline
		Poder negociador de los proveedores              & x   &    &   &   & \\
		\hline
		Intensidad y rivalidad entre empresas existentes &     & x  &   &   & \\
		\hline
	\end{tabular}
\caption{Resumen 5 Fuerzas de Porter}
\end{table}

\item{Análisis Interno}

Aplicando la Cadena de Valor:

\begin{itemize}
\item Actividades primarias
	\begin{itemize}
		\item Logística interna

			Los equipos necesarios para prestar los servicios de sala de ensayo y estudio de grabación no son difíciles de encontrar en el mercado nacional, se tienen bastantes proveedores y una vez adquiridos tienen un costo de mantención bajo. El resto de los servicios ofrecidos tiene más dependencia del capital humano que de los insumos materiales, por ende será de vital importancia mantener un equipo de trabajo estable y con experiencia. Por otra parte un gasto bastante elevado será el de arriendo del local, por tratarse de un sector céntrico de la ciudad. 
		\item Operaciones

			El servicio de estudio de grabación requiere gran asesoría técnica para la banda, por ello será de vital importancia para el servicio contar con el personal adecuado, que tenga experiencia y conocimiento para manejar los equipos y orientar a la banda en el aspecto técnico musical.  El servicio de difusión requerirá personal con conocimientos en publicidad y sobre todo en tecnologías web. El servicio de manager sin duda es uno de lo más dependientes de los contactos que se tengan, más que conocimiento técnico se requiere conocimiento de los principales actores del mundo cultural de la zona, lo mismo ocurre con el servicio de gestión de eventos, que además requiere de un despliegue publicitario.
		\item Marketing y Ventas

			Se hará difusión tanto en la web como en los lugares que frecuentan los músicos de la zona, mediante afiches, avisos de radio, avisos en la web y comisiones por traer nueva clientela. Ofreciendo descuentos a una banda por incorporar a nuevos clientes al servicio.

	\end{itemize}
\item Actividades de apoyo
	\begin{itemize}
		\item Local
		Se necesitará un local que cuente con espacio suficiente para una sala de ensayo, un estudio de grabación, una oficina y un salón de eventos.
		\item Profesionales
		Se debe contar con un equipo con experiencia en grabación, amplificación de sonido y música, no necesariamente con estudios, pero sí con experiencia.
		\item Tecnología
		Se necesitarán equipos que están ampliamente disponibles en el mercado como: Altavoces, amplificadores, instrumentos musicales, etc.
		\item Adquisiciones
		Se deberán tener en consideración las nuevas necesidades de las bandas para ir renovando los equipos acorde a ello, la mayoría o totalidad de los equipos serán adquiridos en el mercado nacional.
	\end{itemize}
\end{itemize}

\item{FODA}

Luego de establecer los puntos obtenidos mediante los análisis 
anteriores, se definen las fortalezas, oportunidades, debilidades y amenazas:

\begin{itemize}
\item Fortalezas
	\begin{itemize}
		\item Ofrecer un servicio novedoso
		\item Conocimiento de tecnología y excelencia en equipamiento
		\item Manejo experto de información, publicidad y contacto con clientes
		\item Equipo de trabajo profesional y comprometido
		\item La concentración de gran parte de los servicios requeridos por una banda en un sólo lugar es la mayor fortaleza del proyecto, esto es lo que marca su diferencia con lo que hay en el mercado.
	\end{itemize}
\item Debilidades
	\begin{itemize}
		\item Costos de inversión inicial elevados
		\item Poseer un servicio que depende de la responsabilidad de otros servicios (eventos por ejemplo)
		\item Similitud de servicios integrados, con los que se ofrecen independientemente
		\item Al prestar muchos servicios se debilita el nivel de especialización en cada uno de ellos, pues los recursos de la empresa se deben repartir para satisfacer las necesidades de cada servicio ofrecido.
	\end{itemize}
\item Oportunidades
	\begin{itemize}
		\item Tendencia creciente de bandas emergentes a nivel regional
		\item Posibilidad de generar vínculos entre servicios similares (salas de ensayo, estudios de grabación, locales de eventos, redes sociales)
		\item Poca competencia regional
		\item Lograr posicionarse como proveedor de bandas para los diversos locales de la zona, obteniendo así más recursos y reconocimiento.	
	\end{itemize}
\item Amenazas
	\begin{itemize}
		\item Posibilidad de alianzas entre servicios a integrar, generando competencia
		\item Las salas de ensayo existentes en el mercado pueden cambiar sus estrategias, aumentando la cantidad de servicios que ofrecen. 
		\item La lealtad de las bandas con sus salas de ensayo actuales.
		\item La posibilidad de las bandas de satisfacer estas necesidades a través de sus propios medios (Ej. Sala de ensayo casera).
	\end{itemize}
\end{itemize}


\item{Selección Segmento Objetivo}
% TO DO
% \red{- falta mejorar la segmentación de los dos segmentos seleccionados, a
%través de los tipos vistos.}

El mercado musical siempre ha sido un tema de interés dentro de lo que es la población adulto-joven de Chile,
en especial dentro de la región de Valparaíso, debido a la gran afluencia de gente en la vida nocturna,
la cantidad de universidades e institutos de educación superior en la región,
dando como consecuencia un flujo creciente de gente interesada en la música.

Algo que marca lo anteriormente dicho, son los variados eventos durante el año que tienen como objetivo
difundir tanto la musica local, como de todo el país. Evento como el ``Rockodromo'', ``Rock Carnaza'',
organizados por las ``Escuelas del Rock''~\footnote{\url{http://www.escuelasderock.cl/}},
los carnavales culturales, etc, los cuales captan la atención y participación de jóvenes como
adultos.

Basado en los estudios del último CENSO realizado el año 2002,
es posible tener una estimación de la relación que existe entre los grupos socioeconómicos
en la V región, donde se puede notar la mayor cantidad de habitantes en la zona, por grupo.

Es por lo anteriormente señalado, y considerando una estimación actual de la población
en la V región, que se ha decidido realizar el presente proyecto enfocado en las personas
mayores de 18 años, pertenecientes a los siguientes grupos socioeconómicos:

Se considera personas del nivel C2, que poseerán cierta cantidad de recursos que puedan
ir enfocados a un pasatiempo, como el desarrollo musical o por que no, que su
objetivo principal sea su banda. Las personas de este nivel estarán habilitadas 
para poder pagar un servicio completo,
y poder confiar en \emph{Music Labs}, para ser asesorados en todo sentido,
dejando como única preocupación, poder madurar su banda musicalmente
y pasarlo bien.

Si bien las personas del nivel C3, no poseen un sueldo lo suficientemente elevado
para poder guardar una suma considerable y dedicarla a gastos de pasatiempos,
por los que se piensa que el presupuesto con el que cuente una banda de nivel C3,
no será capaz de financiar un servicio completo, sin embargo, podrá acceder
a servicios particulares y optar a un sistema de descuentos por inscripciones,
pues \emph{Music Labs} piensa en las dificultades económicas que puedan
tener los posibles clientes.

Se descarta los grupos socioeconómicos ABC1 y D por no presentar un grupo representativo
en las personas mayores de 18 años de la región y por no estar acorde
a los intereses del producto, los que van destinados a bandas emergentes
y sin grandes recursos. El grupo socioeconómico E se descarta por 
no presentan ingresos necesarios para costear un servicio de este tipo.

\item{Ventaja Competitiva}

La principal ventaja competitiva de Music Labs es la integración de muchos servicios, 
que actualmente se adquieren por separado, en uno solo. 

 Al acceder a cada servicio de manera independiente, las bandas necesitan disponer de tiempo para sincronizar
su acceso a horas de ensayo, grabación y difusión, a través de distintos proveedores. Este problema se
mitiga considerablemente si adquieren los servicios integrados, por lo que la ventaja
competitiva implica un ahorro tanto de recursos, como de tiempo para las bandas que prefieran Music Labs.


\item{Estrategias}

\begin{itemize}
\item Estrategias genéricas

Se utilizará una estrategia de diferenciación, debido a que el elemento 
distintivo es la integración de los servicios musicales de sala de ensayo, 
estudio de grabación, difusión y gestión de eventos. La idea es potenciar el uso 
del servicio por parte de las bandas emergentes, con el fin de que se aprecie la 
facilidad que se ofrece, para la creación y difusión de música.

\item Estrategias de crecimiento

Se plantea la utilización de una estrategia de crecimiento horizontal, con 
el fin de comprar los servicios que se ofrecen actualmente de forma independiente,
 pero que se integrarán a la gestión y difusión de las bandas, para así ampliar 
las posibles salas de ensayo, estudios de grabación y demases, reduciendo la
 competencia y aumentar la cobertura geográfica.

\item Estrategias competitivas

Debido a las características del proyecto, entendiendo que buscamos un nuevo espacio
en el mercado para poder tener una posición dominando, es que se piensa en una estrategia
de especialista, pues el proyecto se concentrará en un segmento determinado del mercado,
para obtener el mayor potencial posible.

\end{itemize}

\item{Posicionamiento}

Establecer a Music Labs como el centro de integración, gestión y lanzamiento para todas las bandas emergentes de la región de Valparaíso, para lograr que el servicio sea la plataforma ideal al servicio de los interesados, generando una identificación regional sobre la proliferación de músicos a nivel nacional, facilitando de esta manera la facilidad de gestión de eventos a través de la proliferación de visitantes a la región.

\end{itemize}

\subsubsection{Marketing Operativo}

\begin{itemize}

\item{Producto}


\begin{itemize}
\item Producto Medular

Básicamente, el producto medular consiste en los servicios ofrecidos por \emph{Music Labs}, 
los cuales son: Estudio de Grabación, Sala de Ensayo, Posicionamiento Web 2.0 y Producción de Eventos. 
El primero de estos servicios consiste en la posibilidad de grabación de discos por parte de una
banda, con el equipamiento que esto requiere. El servicio de Sala de Ensayo permite que una banda 
realice los ensayos de sus propias canciones, para lo cual, \emph{Music Labs} deberá contar con los 
implementos necesarios que apoyen este servicio, tales como algunos instrumentos básicos (batería, guitarra, 
teclado, bajo, micrófono, etc), además de su amplificación. El posicionamiento web corresponde a la 
opción de que las bandas se hagan conocidas a través de las redes sociales, y principales páginas a las 
que, como empresa, se tiene acceso. Finalmente, la producción de eventos consiste básicamente en el apoyo
a las bandas, en el momento en que deseen darse a conocer mediante tocatas masivas, o eventos más pequeños.

\item Producto Formal

	Una sala de ensayo con instrumentos y amplificación, dando la posibilidad de ensayar sin instrumentos propios.

	Una sala de grabación con instrumentos, amplificación y asesoría de un experto.

	Un salón de eventos incluyendo difusión para dar a conocer banda.

	Managers para que la banda pueda sacar el máximo provecho a sus presentaciones.

	Publicidad por diversos medios para la banda, incluyendo sitio web y redes sociales.

    \begin{itemize}
        \item Slogan: \emph{``tu opción a la fama!''}
        \item Logo: \\
            \begin{center}
                    \includegraphics[width=0.4\textwidth]{img/logo}
            \end{center}
        \item Explicación nombre: \emph{Music Labs}, proviene de las dos ideas centrales del proyecto.
              \emph{Music}, está basado en el objetivo del proyecto, promover el desarrollo musical en nuestra región.
              \emph{Labs}, es la abreviación de ``Laboratories'',  lo cual es el concepto de todos los elementos y servicios
              que entregaremos para cumplir los objetivos. 
    \end{itemize}
	
\item Producto Aumentado


    Considerando tanto el conjunto de servicios,
    mas la posibilidad de guiarlos para su desarrollo musical,
    lo que podría influir en la fama de su banda,
    se logra una condición de madurez en el mismo mercado,
    siendo del agrado de consumidores exigientes,
    que deseen mucho mas que el simple servicio.

\end{itemize}

\item{Precio}

	Para establecer lo que las bandas están dispuestas a pagar por cada servicio se ha recabado información
    tanto en la V región como en la región Metropolitana y se ha construido la siguiente tabla a partir de
    los precios vistos en el mercado:

    \begin{table}[h]
    \centering
    	\begin{tabular}{|c|c|}
    		\hline
    		Servicio & Precio\\
    		\hline
    		Estudio de grabación &	entre 15000 y 25000 por hora\\
    		\hline
    		Grabación de un tema & entre 10000 y 40000\\
    		\hline
    		Sala de ensayo & entre 2000 y 7000 por hora \\
    		\hline
    		Sitio web & entre 5000 y 20000 mensual \\
    		\hline
    	\end{tabular}
    \caption{Precios}
    \end{table}

    Los precios fueron obtenidos mediante una comparación de lo que ofrece actualmente
    en la región Metropolitana y la región de Valparaíso, considerando
    disminuir los precios, para llamar la atención de los consumedores.

    Los servicios de gestión de eventos y Manager no tienen un valor fijo porque dependen del dinero
    que se recaude en los eventos, en general tienen un valor porcentual a convenir con la banda.

\item{Canal de Distribución}

    Los canales de distribución pensados para el proyecto van a utilizar instituciones intermediarias,
    sólo en el caso de actividades que lo ameriten, como por ejemplo, la compra de materiales para
    el proyecto, como instrumentos, insumos generales, etc.
   
    De la misma forma, los transportes necesarios, también serán llevados a cabo mediante entes
    intermedios,  

    Finalmente, los servicios ofrecidos por \emph{Music Labs}, serán entregados directamente
    a los clientes, por lo que no existiran intermediarios en este proceso.

\item{Publicidad}

	Difusión mediante avisos y afiches en pubs, discotecs y lugares frecuentados por músicos.
    Además se contará con una gran difusión en la web, por tratarse de una las especialidades de la empresa. 

    La publicidad de \emph{Music Labs} no deber masiva, debido a que no todas las personas
    en la región van a estar interesadas en contratar este tipo de servicios,
    por lo cual se centrará en lugares como bares, pubs, discotecs, etc y en horarios
    nocturnos, debido a que la vida musica de la región tiene mayor resplandor
    por las noches.

    De la misma forma, se intentará visitar lugares adicionales frecuentados por musicos,
    como tiendas, centros artísticos, para ofrecer los servicios.

    Finalmente,
    otro tipo de publicidad será llevado a cabo mediante las redes sociales,
    sobre todo MySpace, Facebook y Twitter, debido a la gran cantidad de bandas
    que utilizan estos medios para potenciar su trabajo.

\item{Promoción}

    La promoción principal de nuestro producto, se basará en ofrecer packs promocionales para los nuevos
    clientes, apuntando a ofrecer bajos precios para así captar más clientela, descuentos desde 20\% a 40\%.

    Dichas promociones, serán valida por el primer mes que el servicio sea lanzado al mercado,
    otorgando la posibilidad de firmar contratos por tiempos desde 6 meses a 2 años.

    Adicionalmente se repartirá merchandising acorde a \emph{Music Labs},
    como adhesivos, lápices, llaveros, etc.

\end{itemize}

\subsubsection{Tabla resumen de egresos del análisis de comercialización}		
\vspace{0.5cm}
\begin{table}[h!]
\centering
	\begin{tabular}{|c|c|}
	\hline
	Elemento de publicidad          & Costo          \\ \hline
	Diseño flyers y volantes        & 20.000 mensual \\ 
	Impresión flyers y volantes     & 15.000 mensual \\ 
	Dominio sitio web               & 9.450 anual    \\ 
	Diseño sitio web                & 100.000        \\ \hline
	{\bf Total anual}               & 529.450        \\ \hline
	\end{tabular}
\caption{Tabla de egresos análisis de comercialización}
\end{table}


La publicidad online en redes sociales será realizada por personal de \emph{Music Labs} y no es una actividad que demande mucho tiempo.


