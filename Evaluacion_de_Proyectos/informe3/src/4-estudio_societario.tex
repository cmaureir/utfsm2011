\section{Estudio Societario}
%\red{RUDYAR + RFERNAND}
%rfernand
%rudyar
Es importante definir un tipo de sociedad reconocida por el estado para la
organización de la empresa. Esto determinará la forma en que la organización
pueda crecer y desarrollarse, siendo una elección muy importante al momento de
constituir legalmente la empresa.

\subsection{Tipo de Sociedad}
%rfernand
%°rudyar
La empresa se constituirá como una Sociedad por Acciones, definida por la Ley
N\r{\ }20.190, la cual fue publicada en el Diario Oficial con
fecha 05.06.2007, e incorporada en el Código de Comercio en sus artículos 424 y
siguientes. Esto es dado a los múltiples beneficios para la organización que
ésta ofrece en comparación con los otros tipos de sociedades.

Estas sociedades a diferencia de las demás pueden ser unipersonales, su nombre
debe concluir con la expresión SpA. y su objeto es siempre mercantil. Su
capital se divide en acciones y los accionistas responden hasta el monto de sus
respectivos aportes.

Su administración se puede establecer libremente en los estatutos, es decir,
puede administrarla una persona natural, una sociedad, un directorio, la junta
de accionistas, etc.

Estas sociedades se rigen supletoriamente por las normas de las sociedades
anónimas cerradas.

Entre los beneficios de la sociedad por acciones se encuentran:

\begin{itemize}
\item Se puede constituir por un solo socio, no requiriendo dos personas para comenzar.
\item No se disuelve por bajar su número de socios a uno por la concentración de las acciones en una persona.
\item La transferencia de propiedad parcial o total se realiza por el intercambio de acciones no requiriendo modificaciones al pacto social.
\item Los accionistas (propietarios) de la Sociedad por Acciones no son responsables en caso de que se demande a la sociedades
\item La responsabilidad de los propietarios está limitada al monto que invirtieron en su participación accionaria
\item La Sociedad por Acciones puede poseer bienes, demandar y ser demandada debido a su condición de entidad jurídica separada.
\end{itemize}

Como es posible apreciar, este tipo de sociedades rescata lo mejor de la sociedad de responsabilidad
limitada y la sociedad anónima debido a que las sociedades de responsabilidad limitada son fáciles de administrar, 
pero no existe libertad de salida, ya que la cesión de derechos requiere el consentimiento de los demás socios. 
Las sociedades anónimas presentan un sistema de administración más complejo, pero libertad para transferir las acciones a terceras personas.

De todas las sociedades estudiadas, la sociedad por acciones 
es la que presenta mayor flexibilidad para este proyecto, ya que 
se simplifican los procesos de constitución, administración y modificaciones a los estatutos como los aumentos de capital
y la entrada de nuevos accionistas en el futuro.
 
\subsection{Pasos formación sociedad}
%rfernand
%rudyar
Esta clase de sociedad se puede formar por  escritura pública o por instrumento privado reducido a escritura pública. Además se deberá inscribir el extracto de la escritura pública en el Registro de Comercio correspondiente al
domicilio de la sociedad, y termina con la publicación del extracto en el
Diario Oficial, lo cual deberá realizarse una única vez y dentro del plazo de
30 días desde la escritura de constitución.


%Constitución: Es solemne ya que requiere de acto de constitución social
%escrito. Deberá además efectuarse la inscripción en extracto del acto de
%constitución, autorizado por el notario respectivo, en el Registro de Comercio
%del domicilio de la sociedad. Por último requiere de publicación del extracto
%en el Diario Oficial por una sola vez. La inscripción y publicación deberán
%realizarse dentro del plazo de un mes desde la fecha del acto de constitución.

%Según la tabla resumen:
%No necesita constitución por escritura pública, basta una escritura privada
%protocolizada ante notario (reduciendo costos).

\paragraph{Validación de minuta del acto de constitución social ante notario}

	La minuta se debe presentar ante un notario para la confección de la escritura pública.

    
\paragraph{Registro del extracto en el Registro de Comercio correspondiente al domicilio de la
    sociedad}
		Es recomendable pagar a un notario por efectuar el registro.
\paragraph{Publicación del extracto en el Diario Oficial}
		Es recomendable pagar a un notario por efectuar la publicación.

    El extracto publicado deberá expresar:
    \begin{enumerate}
        \item  El nombre de la sociedad
        \item El nombre de los accionistas concurrentes al instrumento de
        constitución.
        \item El objeto social.
        \item El monto a que asciende el capital suscrito y pagado de la
        sociedad.
        \item La fecha de otorgamiento, el nombre y domicilio del notario que
        autorizó la escritura o que protocolizó el instrumento privado de
        constitución que se extracta, así como el registro y número de rol o
        folio en que se ha protocolizado dicho documento.
    \end{enumerate}

\paragraph{Registrar empresa en impuestos internos}
		
		Llenar formulario de iniciación de actividades y obtener RUT de la sociedad. En el formulario se debe especificar el giro de la empresa:

	221300 EDICIÓN DE GRABACIONES
	

\subsection{Costo de formación de Sociedad}
Debido a un cambio de ley, la publicación del extracto en el diario oficial es gratuita, a menos de que el capital inicial exceda las 5000 UF, en tal caso la publicación tiene un valor de 1 UTM.
%rfernand
%rudyar

Los costos asociados son los siguientes:
\begin{table}[htbc!]
\centering
\begin{tabular}{|p{8cm}|c|c|}
\hline
\textbf{Trámite}                                                                                & \textbf{Duración} & \textbf{Costo Asociado} \\
\hline
Asesoría de un abogado para elaborar y firmar minuta                                            & Aprox. 30 días    & 1\% del capital         \\
\hline
Validación de minuta del acto de constitución social ante notario                               & Máx. 2 días       & 2\% del capital         \\
\hline
Inscripción del extracto en el Registro de Comercio correspondiente al domicilio de la sociedad & Máx. 2 días       & 0.2\% del capital       \\
\hline
Publicación del extracto  en el Diario Oficial                                                  & Máx. 2 días       & 0                       \\
\hline
\end{tabular}
\caption{Costos de formación de una Sociedad por Acciones}
\end{table}


