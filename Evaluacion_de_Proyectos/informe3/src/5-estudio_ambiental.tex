\section{Estudio Ambiental}

La ley 19.300 establece la normativa medio 
ambiental sobre los proyectos que pueden ser 
sometidos al Sistema de Evaluación de Impacto
 Ambiental. \emph{Music Labs} presenta un desarrollo 
de carácter de servicio, en el cual se hace 
utilización de servicios que ya se encuentran
 normados, como lo son el suministro eléctrico, 
servicio de agua potable, servicio de Internet,
 etc, los cuales no presentan modificación alguna
 para su utilización.

Todos los proyectos que se encuentran especificados 
en dicha ley, los cuales deben ser sometidos bajo dicha
 evaluación, no incluyen al proyecto propuesto por \emph{Music
 Labs}, ya que se describen proyectos que afectan directamente al medio
ambiente. Por otro lado los proyectos que
 deben desarrollar un Estudio de Impacto Ambiental, deben 
presentar características relacionadas con emisión de residuos 
contaminantes, alteración significativa del medio ambiente o 
del patrimonio cultural, además de efectos adversos significativos
 sobre la cantidad y calidad de los recursos naturales renovables
 en dicho lugar donde se lleve a cabo el proyecto.

Al momento de definir contaminante (Titulo I, Artículo 2, apartado d),
 se refiere a todo elemento, compuesto, sustancia, derivado químico o 
biológico, energía, radiación, vibración, ruido o combinación de ellos,
 cuya presencia en el ambiente, en ciertos niveles, concentraciones 
o períodos de tiempo, pueda constituir un riesgo a la salud de las 
personas, a la calidad de vida de la población, a la preservación de
 la naturaleza o la conservación del patrimonio ambiental. \emph{Music Labs}
 presenta un servicio musical, que está completamente ligado, si se
 considerase el ruido en ciertos periodos de tiempo.

Lo señalado en el decreto 146, donde se establece la norma de emisión
 de ruidos molestos generados por fuentes fijas, define todo lo
 relacionado la emisión de dichos ruidos, tipos de ruidos, receptor
 de aquel ruido (persona afectada por el ruido), zonas donde se establece
 si es permitida la emisión de ruido por, además de períodos de tiempo y
 niveles de ruido (decibeles) máximos. Se describe la Zona I, aquella
 zona de uso habitacional y equipamiento a escala vecinal, el cual
 está relacionado con \emph{Music Labs}.

\begin{table}[h]
\centering
	\begin{tabular}{|l|l|l|}
	\hline
    \backslashbox{Zona}{Horario} & De 7 a 21 Hrs. & De 21 a 7 Hrs.\\ \hline
	I & 55             & 45 \\ \hline
	\end{tabular}
\caption{Niveles Máximos Permisibles de Presión Sonora Corregidos (NPC) en dB (A) Lento}
\end{table}

En aquel decreto se establecen las especificaciones técnicas al 
momento de medir la emisión de ruido, ya sean externas o internas,
 además de las condiciones en las que debe encontrarse el lugar.

Con lo que respecta a las salas de ensayo y estudio de grabación, por
 el hecho de que aquellos servicios necesitan de manera implícita una
 correcta aislación del y hacia el exterior, por asuntos de captación
 de las ondas sonoras, relacionado directamente con la calidad, se
 puede establecer que el receptor, al cual se define como la persona
 a la cual pudiese molestar el ruido, se encuentra en las lejanías del lugar.

Con respecto a los eventos, se realizaran en el mismo lugar, por lo 
que la construcción de las instalaciones estarán acorde a lo establecido en el decreto
anteriormente descrito, tomando como base los horarios de funcionamiento establecidos
en la Zona I.
