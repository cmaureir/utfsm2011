\section{Estudio Tributario}
%\red{CRISTIAN}

\emph{Music Labs} de acuerdo a los distintos servicios que presta
debe ser fiscalizado por el Servicio de Impuestos Internos (SII)
y regirse a la Ley que corresponda.

Según este ente fiscalizador \emph{Music Labs} está afecta a los siguientes impuestos:

\begin{itemize}

	\item {\bf Impuesto al Valor Agregado (IVA):}
		el cuál grava una tasa de $19\%$ sobre el bien adquirido de acuerdo a lo
		que establece la ley.
		Con respecto a los elementos del sistema de \emph{Music Labs} los que se ven afectados
		por el IVA son:
		
		\begin{itemize}
			\item Sala de ensayo
			\item Sala de grabación
			\item Sala de eventos
			\item Página Web
		\end{itemize}

		Cabe recalcar que este impuesto además se aplica a las inversiones que incurre
		la empresa para desarrollar y/o concretar el proyecto desde la etapa inicial.

	\item {\bf Impuestos a la Renta:} donde se ven afectadas las personas dentro de ciertos
		elementos del sistema de \emph{Music Labs} las cuales son:

		\begin{itemize}
			\item Asesores (Managers)
			\item Asesores Técnicos
			\item Asesores Musicales
			\item Asesores Web
			\item Personal de Oficina
		\end{itemize}
	
		Este impuesto establece una tasa de un $17\%$ de las utilidades en el período
		anual correspondiente, las cuales serán determinadas mediante planillas,
		contabilidad o contratos.

\end{itemize}

Además se deben considerar ciertos documentos con los cuales debe contar \emph{Music Labs}
para llevar un control de los impuestos a pagar.

En ciertos casos se deberán emitir boletas por los bienes y/o servicios prestados en el
momento de la entrega real o simbólica de las especies, así es el caso de la Asesoría
Web en donde se deberá emitir boleta desde el diseño del logo hasta la completa estructuración
del sitio web si así el cliente lo requiere.

También se deberán considerar la emisión de boletas; si son necesarios,
consultores externos que presten sus servicios a \emph{Music Labs}. 
