\section{Resumen Ejecutivo}

En el presente informe se detallarán distintos ámbitos del proyecto \emph{Music Labs},
en la sección de Análisis organizacional se especifica que será una estructura
vertical debido al grado de complejidad de la empresa, empezando por los dueños,
un nivel posterior el administrador y secretaría, y en último nivel las unidades
de informática, gestión de eventos, finanzas y sonido, que estarán en el mismo
nivel de toma de decisión.

Por otro lado \emph{Music Labs} se guiará rigurosamente por el Código del Trabajo,
el Artículo 7 para la redacción y contenido de los contratos.
Por consiguiente si se da término al contrato se utilizará el artículo 159 que
enumera los distintos motivos posibles, excluyendo el término del contrato por
necesidad de la empresa el cuál es detallado en el artículo 161.

Los trabajadores además recibirán constante capacitación de acuerdo a los artículos
180 al 183.
Las remuneraciones de los trabajadores han sido establecidas mediante
la complejidad del trabajo, el cargo de trabajo y el título del trabajador.

La sección del Marco legal especifica la legislación actual por la que se regirá
\emph{Music Labs} para cumplir a cabalidad con la Leyes Chilenas.
La Jornada de trabajo no excederá las 48 horas semanales (Artículo 22, Código
del Trabajo) y 1 hora para almuerzo, además de las horas extraordinarias como
se detalla en el Código del Trabajo.

Existirán dos tipos de trabajadores: los Full-time a quienes se le pagará mensualmente y su
sueldo será superior al ingreso mínimo mensual, y los Part-time que se les pagará por cada
hora trabajada.
Además cada trabajador deberá acceder a un seguro obligatorio contra
accidentes de trabajo y enfermedades profesionales, así mismo los trabajadores deberán
estar afiliados a una Isapre o Fonasa, descontando los porcentajes correspondientes del
salario imponible del trabajador (Ley 16.744).

Por otro lado, de acuerdo al D.S. 745 el lugar de trabajo deberá cumplir con las
condiciones básicas sanitarias y ambientales detallando condiciones físicas,
de agua potable y servicios higiénicos.

La seguridad también se detalla mediante el Artículo 40 enumerando las condiciones y cantidad
necesarias de extintores disponibles en el lugar de trabajo y el nivel de ruido al cual serán
expuestos los trabajadores (Artículo 64 al 72).
En esta sección además se detalla de la patente comercial que deberá adquirir la empresa en
la municipalidad de Viña del Mar.
Y por último en esta sección, \emph{Music Labs} cumplirá rigurosamente la Ley del consumidor
detallando los servicios contratados, precios y condiciones de contratación, y que además
la publicidad será verídica en cuanto al servicio, precio y garantía.

La sección correspondiente al estudio tributario da a conocer la fiscalización a la cual
se debe someter la empresa por parte del Servicio de Impuestos Internos (SII),
que involucra el Impuesto al Valor Agregado (I.V.A) que fija una tasa del 19\% a la Sala de Ensayo,
Salón de Eventos, Sala Grabación y Página Web; y el Impuesto a la Renta que fija una tasa del 17\%
de las utilidades del período anual correspondiente.

En la sección Estudio Societario se describe \emph{Music Labs} será una Sociedad por Acciones (SpA)
en donde su objetivo es siempre mercantil, dividiendo su capital en acciones y los accionistas
responden hasta el monto de sus respectivos aportes.
Este tipo de sociedad rescata lo mejor de la Sociedad Responsabilidad Limitada y Sociedad Anónima.
Además se detallan en esta sección los pasos que se deben seguir para formar la sociedad y los costos
que esto conlleva.

La sección siguiente contiene lo referente al Estudio Ambiental, en donde el punto principal es
respecto al ruido (Artículo 146) debido a que el lugar físico de \emph{Music Labs} está dentro de la zona
I (tipo residencial), en donde existe un máximo de ruidos que se deberá emitir. Cabe recalcar que debido
 al rubro de la empresa, ésta no será sometida al Sistema de Evaluación
de Impacto Ambiental descrito en la Ley 19.300.

Finalmente en la sección referente a otros estudios que sean atingentes al proyecto, se explica por qué
no es necesario realizar estudios adicionales.
