\section{Estudio Marco Legal}
%[JAVIER]
%[SECCIÓN TERMINADA Y REVISADA ORTOGRAFÍA Y REDACCIÓN]

Es importante verificar, al momento de
evaluar y ejecutar un proyecto, que 
se cumple con la legislación del país.
Para que un proyecto sea factible
debe cumplir a cabalidad con las leyes. 
Para ésto, a continuación, se describirá 
la legislación bajo la cual el proyecto debe 
estar sustentado para poder ser legalmente 
factible, de tal forma de lograr tener
un conocimiento claro al respecto.\\

En primer lugar, parte importante
de toda organización son los
trabajadores. Por esta razón, 
es necesario considerar los 
artículos que el Código del Trabajo
presenta. A continuación se detallan
los principales aspectos del ámbito
del proyecto que se deben considerar.

\subsection*{Jornada de Trabajo}

La jornada de trabajo
no excederá 48 horas semanales
tal como detalla el Artículo 22
del código del trabajo. Ya que
las horas diarias máximas de
trabajo son 10, este será
el límite para la cantidad
de horas normales y extras
de trabajo, según sea necesario.
Las horas extraordinarias estarán
pactadas en el contrato de trabajo
según el artículo 32 del código.
El trabajador contará con 1 hora de
colación, según estipula el 
artículo 34. En cuanto al descanso
semanal estará regulado por el
artículo 35.

\subsection*{Remuneraciones}

La remuneración para el caso de
los trabajadores Full-Time estará
fijada de manera mensual. En cambio,
los trabajadores que sean requeridos
para servicios Part-Time recibirán
remuneración por hora efectiva
de trabajo. El monto mensual
de remuneración será superior
al ingreso mínimo mensual, estando
estipulado en el contrato de trabajo.
En cuanto a la asignación familiar
estará establecida en el contrato.

\subsection*{Ley Número 16.744}
Esta ley establece normas sobre accidentes
del trabajo y enfermedades profesionales.
En primer lugar hay que destacar que todos
los trabajadores contarán con el Seguro
Social obligatorio contra riesgos de Accidentes
del Trabajo y Enfermedades profesionales, según
estipula el artículo 1 y 2 de la ley. Cualquier
situación que ocurra en el trabajo o en el camino
hacia él, será considerada accidente según lo que dicta 
el artículo 5. Las enfermedades profesionales
estarán regidas bajo las indicaciones del artículo 6 y 7.
Las cotizaciones de salud serán realizadas por
la empresa según los términos que establece la Ley
en los artículos 17 y 18. El trabajador deberá
estar afiliado a una Isapre o Fonasa según estime
conveniente. La imposición mensual en una AFP estará
dada según el porcentaje que estipula la ley, descontándose
del salario imponible del trabajador, de manera similar
a la cotización de salud.

\subsection*{Decreto Supremo 745}
Este decreto establece las condiciones sanitarias y 
ambientales básicas que deberá cumplir todo lugar de
trabajo.\\ %Además establece los límites permisibles de 
%exposición ambiental a agentes químicos y agentes físicos, y
%aquellos límites de tolerancia biológica para trabajadores
%expuestos a riesgo ocupacional.


Las condiciones físicas del lugar de trabajo cumplirán
con la especificación entregada en los artículos 4 al 9.
Los servicios básicos como son el agua potable estarán
normados por los artículos 11, 12, 13. Los servicios
higiénicos deben contar con las condiciones mínimas
que estipula el artículo 20, 21, 22, 24. Es decir, serán
baños separados para hombre y mujer, con el número de 
artefactos necesarios según la cantidad de gente que
estará trabajando en el lugar.\\

Todos los artículos de trabajo, equipamiento de sonido, 
instalaciones eléctricas y de sonido deben ser mantenidas
para contar con las condiciones de seguridad establecidas
por la norma, las cuales se detallan en el artículo 32.\\

Tal como detalla el artículo 40 del decreto, el lugar de 
trabajo tendrá la cantidad suficiente de extintores. La 
capacidad de cada uno está detallada en el artículo 41. Estos
extintores estarán ubicados en lugares de fácil acceso y con
clara identificación, a no más de 23 metros del lugar habitual
de trabajo de algún trabajador, altura máxima 1.30 metros, desde
el suelo a la base del extintor. Todos los trabajadores
estarán capacitados para usarlos en caso de emergencia. Los
extintores serán mantenidos periódicamente según estipula
el decreto.\\

Debido a la naturaleza del negocio, los artículos 64 al 72 
son de vital importancia ya que especifican la cantidad
de nivel de ruido al que pueden estar expuestos los
trabajadores. En el caso de la empresa, los ingenieros en
sonido, técnicos en sonido deberán estar protegidos de
niveles excesivos de ruido según se detalla a continuación.
Un trabajador no puede estar expuesto a ruido \emph{continuo} en
una jornada de trabajo de 8 horas a una presión sonora
mayor a 85 decibles. Se permiten niveles de presión sonora
superiores a 85 decibles en la medida que no excedan los
valores indicados en la siguiente tabla:

\begin{table}[!h]
 \tiny
\centering
\begin{tabular}{|c|c|}
  \hline
  \textbf{Nivel de Presión Sonora en dB} & \textbf{Tiempo Máximo de Exposición por Horas}\\
  \hline
  85 & 8,00 \\ 
  86 & 6,97 \\
  87 & 6,06 \\
  88 & 5,28 \\
  89 & 4,60 \\
  90 & 4,00 \\
  91 & 3,48 \\
  92 & 3,03 \\
  93 & 2,64 \\
  94 & 2,30 \\
  95 & 2,00 \\
  96 & 1,74 \\
  97 & 1,52 \\
  98 & 1,32 \\
  99 & 1,14 \\
  100 & 1,00 \\
  101 & 0,87 \\
  102 & 0,76 \\
  103 & 0,66 \\
  104 & 0,57 \\
  105 & 0,50 \\
  106 & 0,44 \\
  107 & 0,38 \\
  108 & 0,33 \\
  109 & 0,29 \\
  110 & 0,25 \\
  111 & 0,22 \\
  112 & 0,19 \\
  113 & 0,17 \\
  114 & 0,14 \\
  115 & 0,125 \\
\hline
\end{tabular}
\caption{Niveles máximos de Presión Sonora continua según cantidad de horas de trabajo. Fuente: Decreto 745 del Ministerio de Salud, Gobierno de Chile.}
\end{table}

Respecto a la tabla anterior es importante destacar
que los trabajadores de la empresa no estarán
sometidos a un nivel de Presión Sonora continuo, por
el contrario será esporádico, ya que el trabajo
a realizar es por algunas horas durante el día. Por
ejemplo, al momento de ensayar una banda, una vez
realizada la implementación técnica para el funcionamiento
adecuado de los instrumentos y equipos a utilizar
no será necesario que un Ingeniero o Técnico se
encuentre en el lugar, de ser así, este utilizará
la protección necesaria, tal como se explica más adelante.
En el caso de los eventos, siempre se utilizará
las protecciones auditivas necesarias, debido a
que la presión sonora es más elevada que la normal
generada en un ensayo o grabación.\\

Como señala el artículo 69 será necesaria
la utilización de protección auditiva para
los ingenieros y técnicos de sonido al momento
de trabajar tanto en grabaciones como en ensayos
o eventos.

\subsection*{Patentes Comerciales}
Por el tipo de negocio que se está evaluando
el tipo de patente debe ser comercial. Para
este fin es necesario hacer el trámite en el
Departamento de Rentas Municipales de la
municipalidad correspondiente, en este caso
según el lugar seleccionado para funcionamiento
de la empresa corresponde a la Ilustre Municipalidad
de Viña del Mar. La documentación requerida
es la siguiente\footnote{Información obtenida
directamente de la entidad municipal correspondiente.}:

\begin{itemize}
  \item Fotocopia del RUT del responsable.
  \item Fotocopia de Arriendo o Escritura del local donde se establecerá la empresa.
  \item Informe Sanitario del Servicio de Salud, Certificado Actividad Inofensiva y No Molesta. Se obtiene en Calle Quinta 231, Viña del Mar.
  \item Iniciación de Actividades.
  \item Rol Propiedad. Hay que destacar que el destino de la propiedad tiene que ser comercial para estudio de grabación, esto se acredita 
	en el Departamento de Obras Municipales de la I. Municipalidad de Viña del Mar.
  \item Medición de Decibles. A solicitud del Servicio de Salud.
  \item Solicitar carpeta de ingreso de patente comercial en Departamento de Rentas.
\end{itemize}


\subsection*{Ley del Consumidor}
La ley 19496 del Ministerio de Economía, más
conocida como Ley del Consumidor, tiene por objeto 
normar las relaciones entre proveedores y consumidores, 
establecer las infracciones en perjuicio del
consumidor y señalar el procedimiento
aplicable en estas materias.\\

Tal como estipula el artículo 3 en la sección B
la empresa entregará información veraz y oportuna
sobre los servicios contratados, precio y condiciones
de contratación. En la sección D se toca el tema
de la seguridad al momento de entregar el servicio
al cliente. Por esta razón es fundamental mantener
el equipamiento de sonido y la instalación eléctrica
en condiciones óptimas de funcionamiento para reducir
el riesgo de electrocución.\\

En conformidad con el artículo 28 la publicidad de la
empresa será verídica en cuanto presentará adecuadamente
las características del servicio, precio y garantía.

